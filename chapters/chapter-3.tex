\chapter{The Nature of the Technological Process}

Today's revolution in space and weapon systems technology is a result of the revolution in science, notably in physics, of a century ago. The first step was an intellectual breakthrough made during the period when Maxwell, Hertz, and Mach were making their discoveries and led to Einstein's Theory of Special Relativity. These intellectual advances were a breakthrough because they eliminated some of the restrictions imposed on scientific thought by classical principles. By proposing new theories, individual scientists established a new era in science. Several characteristics of this revolution are noted on Chart 5.

\medskip

\begin{mdframed}[nobreak=true, frametitle={CHART 5: The Intellectual Breakthrough}]
    \begin{itemize}
        \item Work of men of genius.
        \item Required two generations before science accepted and understood the implications of their work.
        \item The basic advances were made over 100 years ago.
        \item The discoveries were in the realm of pure science.
        \item The time of the breakthrough was unpredictable.
    \end{itemize}
\end{mdframed}

\medskip

The second step is a process of translating theory into a device that appears to have some usefulness. The essence of invention is the instinctive or intuitive confidence that something should work and the first rough test of whether it will in fact work. We note several characteristics of this step on Chart 6.

\begin{mdframed}[nobreak=true, frametitle={CHART 6: Characteristics of Invention}] 
    \begin{itemize}
        \item A creative art.
        \item Exploits science, and may support science as well.
        \item Invention is in the realm of technology, not pure science.
        \item Invention can be a lengthy process.
    \end{itemize}
\end{mdframed}

The third step toward a breakthrough results from a decision made at the management level, be it in industry or in the military. Such a decision is based on recognition of the potential importance of the invention. The essence of the decision is to allocate resources to translate an invention into a product that is materially useful. In the military this is usually a weapon system, a major component, or a piece of equipment.

The purpose of the decision is to gain an advantage in time or strength over competitors -- in the market in the case of an industrial breakthrough or over potential enemies in the case of a military breakthrough. The decisions and actions of the enemy have an effect on the decision makers who seek to achieve a breakthrough. The characteristics of this third step are shown on Chart 7.

\begin{mdframed}[nobreak=true, frametitle={CHART 7: The Management Breakthrough}] 
    \begin{itemize}
        \item A decision is made based on recognition of the importance of a scientific principle or invention.
        \item The choice has major implications for future capabilities.
        \item The time required for decision is shorter than the time needed for invention.
        \item The decision allocates resources, and usually leads to a production decision.
    \end{itemize}
\end{mdframed}

In the last step the invention chosen by management is developed as a system and produced in appropriate numbers. An essential part of the engineering breakthrough is the advanced development or prototype. The construction of a pilot plant by industry provides the bridge between the breadboard model and full-scale production.
The military have taken several approaches to this aspect of the engineering breakthrough. We have built prototypes of aircraft prior to production. In our development of missiles we telescoped the construction of the prototype and production into a single phase under the concurrency principle. In our space effort we had planned to create building blocks such as Dyna-Soar and Titan III booster before the manned military space program was shut down. 
The distinction is further blurred by development of one-time, unique systems, such as our command, control, communications, and intelligence systems, which have been evolutionary as new devices and systems have been introduced into ongoing networks and command centers. The characteristics of this fourth step are shown on Chart 8.


 \begin{mdframed}[nobreak=true, frametitle={CHART 8: The Engineering Breakthrough}] 
    \begin{itemize}
        \item Exploits the realm of engineering and technology, not science.
        \item It is a deliberate product of technology with a useful purpose in mind.
        \item Success in this stage is the only real addition to capability.
        \item Requires a shorter period than scientific discovery or invention.
    \end{itemize}
\end{mdframed}

This division into steps, into bits and pieces, is for illustrative purposes only. We should recognize that scientists sometimes take on the role of technologists, that technologists have made scientific discoveries, that production may require invention, and that scientists, engineers, and managers participate in the decision process. It would be misleading to try to summarize all the many activities of a multitude of individuals in complex technological relationships in four simple steps. Historical experience is complex and the four steps we have discussed are only indicative of broad areas of human activity.

Also, there is no uniformity in this process. At times, individuals have tried to stimulate closer ties between science and technology; Galileo and Newton, for instance, tried to cross-fertilize these two fields. Diesel's attempt to apply the law of thermodynamics (made possible by the high pressure steam engine) led to the invention of the Diesel engine, but the forecast that it would be the best engine for aircraft was clearly wrong.

However, our interest is in the use of science and technology as elements of strategy and conflict. Let us look at these four steps in this context. The revolution in physics that began with Maxwell and Mach led to new theories, which in turn led to independent work by Fermi and others. By contrast, the atomic bomb was the result of determined effort. The policy breakthrough in this historical example was the decision by the president to spend the large sums of money required to construct a useful weapon. It was based on recognition by the scientific community, notably by Einstein, Wigner, Szilard, and Fermi, of the practical implications of an advance in basic science.

In the case of the ballistic missile the direct relationship between science and weapon is not quite as dramatic and clear-cut as in the example of the atomic bomb, partly because war rockets are ancient. However, Goddard's initial investigations of rocket propulsion and Oberth's theoretical calculations played key roles in the German development of the V-1 pulse-jets and V-2 rockets. Here is an example of an invention being recognized and resources being allocated for an engineering breakthrough. The Germans made this decision in 1932, and chose two different approaches, rockets and jets. The first V-1 and V-2 flew about ten years later.

The German engineering work played a significant role in Russian rocket development and in our own as well. For example, both the Redstone and the Russian T-1 and T-2 used oxygen and alcohol. However, the technical paths diverged at this point. The Russian strategy was to pursue an engineering approach to missile development. We, on the other hand, chose to await an invention in nuclear weaponry to give us a lightweight, high-yield, nuclear bomb. Once this invention had been realized, we made the decision to allocate resources to our missile program and sponsored the many engineering breakthroughs in guidance, airframe construction, and reentry technology required for operational missiles.

In summary, then, we see that the atomic bomb followed the four-step pattern; however, in the missile field the division is not so clear-cut, notably because the policy breakthrough came so late that technology from other areas of research had caught up with missile technology.

In its broadest sense the term technological breakthrough applies to the entire process when it results in advances that thrust us into a new era of military capabilities. However, the term is used also in connection with limited parts of the process. A new theory may be described by scientists as a breakthrough. An inventor may describe his work as a breakthrough. The engineers working on a specific part of the problem of production may describe an advance they make as a breakthrough. This is most likely to occur when an invention is necessary for production; use of the term breakthrough has some validity because without the invention, production would not be feasible or efficient.

The key step in the process is step three, the policy breakthrough. A decision in the realm of the engineering breakthrough cannot be considered in isolation from effort allocated to steps one and two. The importance of the policy breakthrough cannot be overemphasized.

In attempting to bring order and control to the technological breakthrough, we have in the past concentrated on steps three and four in the process. We have studied management and decision procedures in more detail than the intellectual breakthrough and invention. We have brought a great effort to bear on production so that systems are made realities in a minimum of time. We consider it a major breakthrough when the time covered by steps three and four is reduced from eight years to five years. We have not made a similar effort to reduce the total time covered by the entire process.

At present, the period covered by the intellectual breakthrough and the invention cannot be reduced. This is an unavoidable consequence of our scientific and technological effort, partly because steps one and two lie outside the military sector. In their broadest sense steps one and two are the consequence of our society, and our contemporary society has not organized an effort to influence these steps. The way we approach invention has changed: in the past, invention was usually the work of an individual; today, we are making an institutionalized effort to stimulate inventions. But this change is not always productive, because it may stifle the loner and out-of-step creativity.

Once again there is much to be said for both sides of this argument. The team approach is not always superior to an individual approach to an invention. Some creative individuals cannot work as members of a team; others function best as part of a team. Furthermore, some parts of science notably chemistry, seem to require a team effort to make advances; but in the field of physics and mechanical engineering, more advances seem to be made by the individual working alone. On the other hand, there is the difficult task of interdisciplinary work. Regardless of the approach followed, it appears difficult to reduce the time necessary for intellectual breakthroughs or inventions and it is unpromising to organize according to pat formulas.

It may well be that recognition and acceptance of new theories and inventions will always require a period of mellowing, testing, and evaluation. Early dissemination of the new idea would help -- provided its significance is recognized. Some say that new ideas never win by persuasion; they merely take over as their opponents die off. In any event, a new theory usually has little impact within one and often even two decades. This brings us to another aspect of the breakthrough.

From the point of view of technological strategy, our principal concern is in the time when such advances occur. The invention of a new jet engine today would not produce a new era in military capabilities as did the first jet produced by Whittle. Conversely, the invention tomorrow of a practical way of using focused energy beams as weapons would alter radically the whole sphere of military activities. Time is especially crucial in technological maneuver.

Whether the breakthrough is a surprise to the enemy or is an advance that he anticipates but cannot counter, the side making the breakthrough should plan for technological pursuit to maximize the gain made possible by the new advantage. Pursuit has proved difficult in warfare. The losses sustained in winning the battle frequently have reduced the momentum of the winner. Also, uncertainty about the conditions of the loser has made the winner act with caution.

In technological conflict pursuit is facilitated by the circumstances surrounding the breakthrough. Rather than causing losses, the technological success increases the power of the side making the advance, and success often heightens morale. The breakthrough can reduce the amount of uncertainty about the enemy's technology position.

These circumstances point out clearly that significant technical advances must be exploited. The concept of pursuit has a valid role in technological conflict. This is well-illustrated by Soviet space activities. Once they achieved a clear advantage over us in space they engaged in a form of pursuit to negate our attempts to make any advances in this new arena of conflict.

Moreover, this advance was used as the basis for maneuvers in other forms of conflict. In 1961 the Soviets broke the "gentleman's agreement" on testing nuclear weapons in the atmosphere, and then prevented the "neutral" powers from criticizing them. This advantage in another aspect of the technological conflict is an example of technological pursuit.

The full consequences of the Soviet decision to ignore the spirit of the 1972 ABM Treaty and go ahead with ballistic missile defenses while simultaneously improving their ICBM force and greatly increasing its numbers, were not recognized until 1983. The expansion of Soviet ICBM capability may have been one of the most crucial moves of this century. After 1983 the US began the painful process of catching up, but we have not yet done so.

Pursuit is not the exclusive province of the aggressor. The defender should plan on pursuit when he has acquired an advantage over the aggressor. Up to the present, we have yet to engage in pursuit to overcome the Soviets. On the contrary, we have halted short of using our superiority in aeronautics, nuclear weapons, computers, or missiles to cause the Soviets to modify their goals, strategy, or operations.

As the side on the defensive, we have one advantage which comes not from our technological strategy but from our resources. That advantage is mobility. We can change the priorities of our efforts and counter new threats as they appear. The richness of our technology makes this mobility possible.

(This was amply illustrated in the SDI program, where we were able to investigate a number of alternate approaches to ICBM defense. Unfortunately, we have not done as much to exploit these advances as we might.)

The crucial problem is to meet the threat on time. This is especially vital for us because we are on the defensive, have never tried to achieve surprise, and have never engaged in technological pursuit. The Soviets need not be as concerned about the time dimension of technological conflict, since they know that our goal is to maintain the geopolitical status quo and not to overthrow the Soviet \textit{nomenklatura}. Thus, our advances pose threats only to their near-term goals abroad, and never to their security at home.


\section{U.S. Policies and Technological Progress}
As we stated in the last chapter, the United States has no overall policies with regard to technological development. In part this is due to the decentralization of technological resources in independent private industries, and is a benefit to our overall progress. Unfortunately, we have no policy or strategy at the governmental level, although paradoxically we do suffer there from overcentralization of the decision process. However, our central decision makers are not guided by strategic considerations and projects are related to each other mainly through budgetary actions. Various projects have their goals and we make extensive efforts to relate projects to each other, but the relationships do not come from a felt need to execute strategy. Without strategy, there is no mechanism for integrating goals, tasks, and priorities, and there is no criterion for the weighing of risks and costs.

Our technological effort is guided to some degree by conflicting policies. For example, we assert frequently that we are advancing along a board front. Also, we minimize direction and control, for in that way we assist progress. Consequently, innovation and invention are where we find them; we abhor invention on schedule. From the point of view of Protracted Conflict, however, we do not have an integrated technological strategy.

We do have budgetary controls. Each project is made to compete for funds, generally on the basis of the skill of its managers in playing financial and political games and the persuasiveness of its supporters. This is probably inevitable in a democratic society, but the results are sometimes bizarre. Projects are often assigned to different regions for purely political purposes. At one time the U.S. Air Force found its technological resources scattered from Boston (electronics) to San Bernardino, California (ballistic systems), and managers of crucial Air Force space projects still spend as much time on airplanes as they do at work.

These are some of the major restraints we face in regaining the commanding lead we once held. There are others. Some lie in our technology itself: although technological research can be directed and certain lines of research emphasized, there are limits. The first jet could never have been produced in 1900 nor the first atomic bomb in 1915.

Another restraint is the technology base. The space systems now in operation are an outgrowth of our missile technology. We have, in the past 20 years since the first edition of this book, begun to recognize the importance of technological building blocks, and have constructed some of the necessary facilities such as environmental laboratories on earth and underseas, although, except for the very temporary Skylab we failed to build a manned orbital laboratory.

Considerations of strategy impose still another restraint. We must have at all times the in-being force necessary to win wars. This means being ready for operations at every moment in the foreseeable future while providing simultaneously the foundations for major advances in future capabilities. These are requirements that compete for resources. Our in-being capability is not static; we cannot allow it to dwindle or become obsolete. Thus, modernization of our forces must be continuous but it cannot detract from having sufficient power at any given time.

This restraint is compounded by a third restraint, which is financial. There is an upper limit on what we can expend to advance technology in general and on what we can allocate to develop specific systems. For example, no amount of money spent in 1935 would have given us our first ICBM. Unlimited resources in 1950 would not have give us Apollo 11. In attempting to achieve a technological breakthrough we must reckon with restraints imposed by funding.

These restraints have their greatest impact on step three. In the policy breakthrough, the attitude toward technology plays an important role. If decision makers are convinced that advances occur automatically, if they believe that contemporary technology can give us at any moment an unexpected but major advance in military capability, they will be restrained from taking effective action. Such an attitude makes them reluctant to choose a weapon or warfare system to be developed and produced because a breakthrough would make it obsolete and unnecessary.

A belief in millennium tomorrow is based on the unstated assumption that advances come automatically because of the nature of our present environment. From a cursory glance at past breakthroughs it should be apparent that they are the result of deliberate human action, that is, a combination of goals and work to attain goals. Nevertheless, the result of this attitude is a belief that choices are unnecessary because advances are spontaneous.

Another aspect of this step is a seeming paradox. The decision maker, while awaiting a technological breakthrough at any given time, feels he is suffering from an embarrassment of riches. As he faces the choice of a course of action he sees so many ways to proceed that he finds it difficult to choose any one of them. Furthermore, the rate of advance makes him hesitate. For if he chooses, he may soon find that the system selected has been made obsolete before it is usable.

These aspects have important repercussions. The first is that they delay decisions. Secondly, the decision makers press the military planners to examine minutely the entailed decisions which spring from the courses of action possible. Additionally, they press them to forecast with certainty these anticipated effects. A recourse to science is the planner's response to such impossible demands.

Here we should note another paradox in this process. The scientist and technologist are responsible for advances in knowledge and in applications. Authority in these fields does not per se provide insight into what is either commercially or militarily useful. The management level in industry uses scientists for technical advice but does not depend on them for managerial decisions. However, in the military, management procedures are designed to have scientists participate; thus while individual scientists can initiate an advance other scientists can restrain the project.

Past attempts to put more objectivity into our decision making by considering cost-effectiveness and by using computers for war games had only a limited validity. They contain an inherent danger because the results are inevitably biassed, even forced, by the assumptions governing the game. If the simulation designers do not recognize crucial factors, those factors will have no effect on the game results. The main decision is still that of a choice of strategy which, in turn, must reflect an assessment of the enemy's strategy.

Many strategic considerations do not lend themselves to computer simulation, because they cannot take into account all the relevant factors. As an example, in the computerized war game situation the surprise element is usually not considered and, therefore, a basic distortion may be introduced. Modern computers are useful to determine patterns and to help in visualization, but they don't substitute for the strategist.

The challenge is to create and execute a technological strategy. Technology should be the servant of the strategist, who must be a thorough student of strategy and its history.

The weapon system as such is not the goal of technology. The weapon system is the tool of the soldier or of the man carrying out a selected strategy. This is true even of push-button weapons. Conflict occurs between men or between societies.

\section{Technology and the Economic Base}
Technology develops faster than the economic base. This elementary fact prevents us from taking advantage of all technological possibilities. Technology grows according to geometric progression, whereas the resource base grows, at best, according to an algebraic progression. Sometimes it even retrogresses. Included in this resource base is the human factor, and that may not grow at all.

The heart of the matter is not just a question of inventiveness and organization of the scientists or the scientific base. It is the optimal utilization of economic resources and the proper integration of the technological, economic, and strategic resources. This integration is essentially a two-way street. The strategist must be able to request technological solutions for his problems, which can range from space warfare to propaganda. But in turn the technologist must tell the strategist what the potentials and limitations of his strategy will be.

\section{The Technological War General}
In technological warfare, generalship is the key to success, as it always has been in every conflict. The difference today is that generalship on the battlefield is perhaps less important than generalship exercised many years before a battle is joined. This is especially true of the generalship that goes into the design and development of weapon systems. The general who wins the battle is usually the man who held decisive control ten years before the fighting started and who, at the moment of battle, is either dead or retired.

Note that this applies equally to Commanders in Chief, and behind them to the Congress. Andrei Gromyko has met fourteen American Secretaries of State during his tenure as Soviet Foreign Minister. Cyril Korolov commanded the Soviet space program for a considerable period of time. The Soviet tyranny, by its nature, has the advantage of being able to make and keep long range plans. An American President, by contrast, must spend money and make unpopular decisions that bring results during the administration of his successors. The temptation to let the future take care of itself is intense.

Technological generalship must anticipate strategy, tactics, and technological trends. It must develop weapons, equipment and crews. Such developments must be anticipated in advance of trends.

Generalship in battle still is of great significance, especially because of the surprise element in modern war. Here the general must be the man who can get the maximum performance out of the systems he actually possesses. He must have an inventive mind, to carry out modifications that become desirable. If he cannot overcome a technological lead by the opponent he must be able to devise tactics or stratagems to carry out his mission despite technological inferiority. While not necessarily a battle leader -- although battle leaders are still required -- he must be a great thinker. He must have full knowledge of his weapons systems and those of the opponents. Finally, he must be able to think through the lessons of the battle, even as the battle is being fought.

Implicit in this description of generalship is the assumption that the leader is striving to reach selected goals and that he is using initiative in his actions. Our technological strategy of the future must try to take the initiative in a selected field and to defeat the Soviets clearly on as many occasions as possible. For example, there is no reason why guerrilla warfare and counterinsurgency should be their exclusive domain. Technology can make it possible for us to contain them in these forms of conflict as well as in nuclear war. Fortunately, as the lessons of Vietnam were learned we have devoted some attention to the technology of people's war.

\section{Conclusion}
The U.S. goal is to make the Technogical War remain an infinite game; one which will never be "won" in the sense that one side eliminates the other through armed conflict, especially nuclear war.

The challenge is clear. We are engaged in a conflict for technological dominance. The center of our power position is threatened by the Soviet drive to surpass us and become superior. While the relative technological position is important to political, economic, diplomatic, and psychosocial struggle, it is vital to military conflict.

Superiority in military technology is the prerequisite of strategic success. This is especially true in the era of aerospace nuclear warfare, when a surprise attack made possible by an unexpected technological advance could lead to sudden defeat of the seemingly strongest power. The danger is especially acute in the current period when expanding technology can be used to implement aggressive ideology. In spite of the richness of U.S. resources, two resources are neutral: time and will. The time advantage goes to the one who has the will to grasp the initiative.

In order to use time successfully, we need an integrated technological strategy. Such a strategy will require basic changes in American organization and decision-making processes. To survive, we have no choice but to pioneer.

\begin{mdframed}[backgroundcolor=black!10, frametitle={THIS PRINCIPLE HAS NOT CHANGED. [1997]}]
\end{mdframed}
