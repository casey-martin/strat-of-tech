\chapter{Surprise}
\begin{quotation}
    \textit{It never troubles the wolf how many the sheep be.} \\
    --Virgil
\end{quotation}

Surprise has long been a key aspect of war. The history of surprise has been analyzed from the point of view of the surpriser and the surprised, defender and attacker. Many kinds of surprise have been identified: strategic, tactical, operational, and technological.

One inherent element is warning. Warning results from a combination of intelligence and reason; lack of information about the nature and course of events and lack of time in which to take action after a threat is perceived contribute to the devastating effect of surprise.

Surprise in modern war is vastly different from surprise in the past. At the operational level the ballistic missile with intercontinental range and time of flight in minutes; orbiting bombs of the kind developed by the Soviet Union with times of re-entry measured in minutes; space-based sensors which can detect and report events in seconds; and lasers which have almost instantaneous kill over vast distances all have changed and will continue to change the very nature of surprise in war.

Ballistic missiles and space systems have had a dramatic influence on both tactical and strategic surprise. Combinations of sea-based and land-based intermediate and long range ballistic missiles can be used to confuse sensors and overwhelm the data processing systems of the surprised. Conversely, the data from sensors, especially space based sensors, can be correlated to give much more accurate information of events in real time, and thus provide warning of the tactics being employed by the surpriser.

The responses available include launching many missiles simultaneously to saturate the sensors and prevent accurate intelligence on the number of missiles launched, and maneuvering the re-entry vehicles to deceive the surprised as to the targets being attacked.

Space based systems are essential to prevent strategic surprise. They can report events over a prolonged period so that slow and rapid indicators of changes in normal patterns or operations can be interpreted as opening moves in potentially threatening operations. They provide global coverage but also can be directed to cover specific locations anywhere on the surface, in the oceans, in the atmosphere, and in space.

The surpriser must plan to deceive such space based systems (possibly by destroying them) as well as prevent being surprised himself. These systems are especially important in an era of arms control, because they are generally the only reliable way to verify the opponent's compliance.

Prevention of surprise in the modern era demands access to space; anything which prevents access to space enhances the possibility of surprise.

Technological surprise is in principle much harder to achieve than operational surprise because of the long lead time from concept to discovery through development to eventual military application. However, the accelerating rate of change in electronics makes it possible to retrofit the guidance and data processing elements of existing systems and thus achieve much higher-than-expected performance, as for example in accuracy, and thus contribute to surprise. A more subtle form of advance can also lead to surprise. Passive defense measures, such as hardness, deception, and mobility, which are difficult to detect in the R\&D phase can reduce the effectiveness of the attacker.

Unfortunately, defensive surprise, while possibly decisive, is not much use in deterrence of war.

\begin{mdframed}[backgroundcolor=black!10, frametitle={AFTERTHOUGHT FOR THE DAY}]
    \textit{Surprise, when it happens to a government, is likely to be a complicated, diffuse, bureaucratic thing. It includes neglect of responsibility, but also responsibility so poorly defined or so ambiguously delegated that action gets lost. It includes gaps in intelligence, but also intelligence that, like a string of pearls too precious to wear, is too sensitive to give to those who need it. It includes the alarm that fails to work but also the alarm that has gone off so often it has been disconnected . . . It includes the contingencies which occur to no one, but also those that everyone assumes that somebody else has taken care of.} \\
    --Julian Critchey, \textit{Warning and Response}, 1978
\end{mdframed}

\section{The Sneak Attack}
The popular view of surprise in modern war is identified with a sneak attack, that is, operational surprise. Our experience at Pearl Harbor makes it easy to understand this belief, while the widely-known characteristics of the intercontinental ballistic missile permit us to grasp readily the nature of a future surprise ICBM attack. The missile is the ideal weapon for a rapid sneak attack, not just against one base like Pearl Harbor but against entire countries and continents.

Of the characteristics that make the missile suitable for a sneak attack, the most important is speed. The total flight time of an intercontinental ballistic missile from the USSR to the United States is about 30 minutes. Space-based systems could increase the warning of an attack almost to the total missile flight time; but even if we are given this much warning, the intercontinental ballistic missile has changed the dimension of surprise and has given the aggressor a most potent tool.

Without access to space the United States may well find itself blinded at crucial moments.

Even with warning the US can do little other than launch the force in a classic "use them or lose them" scenario. Lack of adequate defense forces the defending power to a doctrine of launch on warning.\footnote{It has always been exceedingly difficult to get arms control advocates to understand this elementary principle: if the retaliatory weapons don’t survive, there can be no retaliation; and if the aggressor knows there won’t be a retaliation, then deterrence is thin to non-existent. Strategic defenses are stabilizing, not destabilizing, because they are dangerous only to the aggressor.

Strategic defenses make strategic offense weapons obsolete. No one in his right mind believes that strategic defenses can form an ‘impenetrable shield’ against a modern technological power like the USSR; thus they are not an incentive to a first strike. This point has been made repeatedly in our advocacy of a policy of "Assured Survival" as opposed to the official US Policy of "Assured Destruction", and appears to be taking hold in some part of the armed services, but not in the State Department.}

\begin{mdframed}[backgroundcolor=black!10]
    The alternative is to accept the damage and try to ride out the attack, then retaliate. This may have been a feasible option in the 60's, but by 1975 the Soviet Union had achieved ICBM accuracies of a few hundreds of feet. No passive basing system can protect missiles against nuclear weapons delivered at those accuracies.
\end{mdframed}

A massive intercontinental ballistic missile attack launched by an aggressor is an ever-present danger. Such an attack would come as the culmination of a series of measures, operations, and techniques, orchestrated to achieve maximum psychological effect on the surprised. The aggressor would have undertaken specialized campaigns in the various elements of conflict -- political, psychological, economic, military and, above all, technological.

Once the time is ripe, the attack comes suddenly and catches the defender asleep. But despite the present concentration on the sneak attack, surprise is not the exclusive province of the aggressor. Defenders have used surprise to great effect in the past and should strive to do so in the future. The future security of the United States requires that our strategy include measures to achieve surprise, as well as those to prevent it. The main surprise to aim for is that we won't be surprised.

Before we examine the broader aspects of surprise, let us point up the fundamental aspects of the sneak attack. First, surprise is tactical. Second, this form of surprise is used by the aggressor, not the defender. Third, it will be achieved only with the most advanced weapons. Fourth, prevention of surprise requires use of the most advanced technical means.

\section{Strategic Surprise}
There are also surprises on the strategic level. For illustrative examples, let us look at two of the ways in which the USSR has actually achieved strategic surprise in the decades since World War II: the opening of the space age, and nuclear testing during the test moratorium. As a result, the Soviets obtained a lead over us in space that has only partially been overcome by our massive and expensive NASA spectaculars. They lead in many military phases of space, whereas we are ahead in nonmilitary uses; in near-earth operations their lead may be as much as three years.

\begin{mdframed}[backgroundcolor=black!10]
The above was written in 1969. Since that time the U.S. has allowed the Soviets to take a commanding lead in near-Earth space technology. The Soviet Mir space station is fully operational, while the US does not intend even to attempt a space station prior to 1992. As we write this in 1988 the Space Station faces increased Congressional resistance, and its funding is in doubt.

In addition, the Soviets developed and deployed an operational satellite destroyer, which was, because of political opposition, not countered with a US anti-satellite weapon. The US satellite interceptor program was deliberately abandoned, although its feasibility had long been demonstrated. The arguments against development of the US anti-satellite weapon were largely based on the arms control theory that we need space assets more than the Soviets; therefore it would be better if neither side had anti-satellite weapons. If we don't build ours, we can hope the Soviets won't build more of theirs.
\end{mdframed}

Few now recall when both the US and the Soviet Union engaged in unrestricted nuclear tests. The US was induced to observe a "gentleman's agreement", that is, an informal ban on nuclear testing. Then, suddenly, the Soviets began a massive series of above-ground tests that included the detonation of the largest hydrogen weapon ever exploded; and followed that with the offer of the Treaty of Moscow banning above-ground tests. The result was that the Soviets gathered a great deal of experimental data denied to the West.

The moratorium allowed the Soviets to determine critical effects of nuclear explosions in space. Because we honored the test ban, we let much of our testing capability atrophy, and now the Treaty of Moscow prevents us from finding out just how far behind we are in the application of nuclear weapons in space. The impact of these surprises cannot be calculated with precision but the Soviets gained a considerable time advantage in offensive orbital weaponry and ballistic missile defenses. Note that preparing for strategic surprise must continue over a period of several years.

These two surprises occurred in the technical phase of the Technological War, not in the military phase. They were achieved by an orchestrated strategy that employed several forms of conflict, including intelligence operations, propaganda and psychological warfare, political and diplomatic maneuverings, and a concentrated technical effort. While the goal of the Soviets has been to develop advanced weapon systems, such weapons were not employed militarily in these two surprises; however, military technology was developed, and diplomacy and treaties closed off our access to the means of catching up or at least made it difficult.

The best way to counter surprise is to deploy the most advanced technology possible and continue to modernize the strategic forces. This is not to imply that the technical effort must be devoted exclusively or even oriented primarily to countering potential technical surprise; but as we have insisted, surprise must be made a key element of any technological strategy. Since technology has given a new dimension to surprise in the strategic equation, technology is needed to support our own or prevent enemy surprise in all forms of conflict.

The misconception that surprise aids only the aggressor -- a misconception that stems from thinking of surprise only as a 'sneak attack' -- is especially harmful in the Technological War. In his classic work on surprise, General Erfuth\footnote{General Waldemar Erfurth, Surprise, S. T. Possony and Daniel Vilfroy, translators. Harrisburg: Military Service Publishing Co (Stackpole) 1943.} has shown that there are two parties to the operation, the surpriser and the surprised -- this is not the same as saying the attacker and the defender. The defender also can employ the technique of surprise, and perhaps more effectively than the attacker.

Furthermore, there is a widespread misunderstanding that surprise refers exclusively to the initiation of war. Some writers consider surprise to be just a more elegant term than sneak attack. To other writers, surprise is tantamount to technological surprise. This is far too restrictive an understanding of surprise and its role in modern war.

\section{Tactical Surprise}
Tactical surprise is essentially surprise in combat. It is used to prevent the enemy from bringing adequate forces into operation in time to counter those used against him. The weapons of the surpriser are used to bypass or neutralize those of the surprised. Without surprise, the attacker would be required to use massive superiority to crush his opponent. The difference is like that between judo and a bare-knuckle fight.

Tactical surprise usually does not lead to the nullification of all of the opponent's armament, but if it is well-conceived and backed by technological improvements and adequate forces tactical surprise can go a long way toward eliminating enemy weapons as a relevant factor. Given the complexity of modern systems, the surprised opponent is faced with considerable delay before he can readjust his tactics; in a fast-moving war such readjustment may not be feasible.

Under modern circumstances time and technology as well as combat procedures are needed to gain tactical surprise. Technology can produce new types of weapons, new weapon effects, improved weapon effects, improvements in delivery systems, combinations of weapon systems, better active defense, and so on. Examples ranging from the "War of the Iron Ramrod" of Frederic the Great to the devastating effect of Lee's rifle pits at Cold Harbor show that technology and its proper tactical use may achieve surprise. With superior armaments or doctrines, and with troops trained in their use, the entire armament of the opponent can be nullified.

While this is the ultimate goal of tactical surprise, it is usually difficult to achieve. This is so because the possibilities of complete technical surprise are limited. Because of time required to develop a new weapon system, opportunities are increased for technical warning and for counterefforts, either technical or operational. Furthermore, excessive secrecy or failure to deploy weapons can result in surprising one's own troops, with disastrous results -- as happened with the use of the mitrailleuse by the French in 1871. On the other hand, tactical surprise can be accomplished by a minor weapon improvement that from a technological point of view may be marginal but which today or tomorrow may facilitate victory in battle by creating a decisive advantage.

\section{Strategic Surprise through Operational Surprise}
Surprise can result from operations of the forces available, as well as from technological innovation. To achieve surprise of this type, the commander operates in a way unexpected by his adversary; in the ideal situation the enemy is unable to devise countermeasures in time. The attacker hits the defender where and when he does not expect to be hit.\footnote{One clear example of this kind of surprise was the Fall of France in 1940. Not only did the Germans attack in a place thought totally unsuitable for armor, but they used their armor in unexpected ways, driving deep into the French interior without waiting for the infantry to catch up. They also used their aircraft as long range artillery to neutralize the French artillery which had been placed so as to be out of range of German artillery but able to bombard any attempted river crossing. Once the river was crossed, the French artillery could be engaged by German infantry and light armor.German armor then penetrated deep into the French interior. The result was the the Germans operated inside the French decision cycle: by the time French headquarters had considered the situation and issued orders, their information about the front was obsolete.}
Or, conversely, the defender reacts by hitting with weapons or with performances the attacker did not anticipate and against which he cannot protect himself properly; the defender counterattacks when and where he is not expected.

The number of operational variations is truly infinite, and the details of such operations usually can be planned and prepared with a high degree of secrecy. These variations are possible because of the multiplicity of weapons, the great spectrum of their performance, and the vast number of operational options.

Opportunities to use operational techniques to achieve surprise arise from various combinations of the performance of the carriers of destructive agents and the effects of those destructive agents when they are transported to the target -- from the possibilities of multiple routes and methods of attack -- from the variety of environments -- and from countless other factors and their combinations. In addition, there are the skills of tactics, the principal one of which is to use a military force in a surprising manner. The use of expedients, saturation, and other techniques that cause uncertainty create further possibilities for operational surprise.

\section{Technology and Surprise}
We repeat, surprise is not confined to active combat. Even though hostilities are not occurring now, the battle for tactical advantage and the effort to achieve surprise goes on incessantly. Laboratory is pitted against laboratory to find new advances such as radar techniques for looking over the horizon and for distinguishing between warheads and decoys. The laboratories struggle to compress data so that information, particularly details on attack, can be instantaneously transmitted and presented to decision makers. They search for new concepts that can find expression in hardware and tactics.

In addition, there is the broad area of strategic deception in matters of science. This includes deception about the general state of excellence, the level of progress in a given aspect of science, and the application of science to specific weapon and component development. It seems that behind the Iron Curtain there is a second curtain that conceals the nature of Soviet science.

To conduct this deception, the Soviets release scientific articles and withhold others, thus creating a false impression of their successes, failures, and interests. Another method is to send scientists to international meetings, where they either spread misinformation or are evaluated by their counterparts as not being knowledgeable or as being geniuses. Such evaluations may lead to all kinds of false deductions.

For example, during the test-ban debates we saw arguments that the Soviets did not know anything about decoupling techniques to conduct nuclear tests underground in secrecy. Also, we were told by Soviet leaders that the day of the heavy bomber had passed -- which did not deceive us. On the other hand, we were quite surprised when the Soviets sent a man into space, although they had been forewarning us; and their recent exploits in space, including the Mir space station, and the "Red Shuttle" took many of our decision makers by surprise.

\section{Stratagems to Achieve Surprise}
Scientific deception can have a great impact on research and development lead time. The United States has devoted a great deal of effort to reducing the time required to translate a scientific theory, discovery or invention into a practical weapon system. In spite of much study we have not reduced the time interval to less than five years. (Since that was written the procurement time has grown from five years to ten and more.) To develop and produce a weapon in even this lengthy time costs billions of dollars, and the long lead times reduce the prospects of achieving surprise.

Scientific deception aims at keeping the enemy's lead time as long as possible. In this way a significant military advantage may result. This advantage may be crucial at the tactical and operational levels where it could have a direct impact on a strategic decision such as overt aggression.

The ultimate goal is to gain a strategic advantage by acquiring a major new family of weapons while concealing from the enemy that it is being developed. The appearance of a brand new weapon often is termed a breakthrough. When a nation makes a breakthrough of this type, as we did with the atom bomb, the British with radar, the Soviets in space, an entirely new arena for military operations is opened up. If the breakthrough leads to a military advantage that the enemy cannot counter in time, such as domination of the air, space, or deep water, the breakthrough may be decisive.
\\
Strategic surprises can be accomplished in many ways. A few examples are:
\begin{itemize}
    \item The choice of a strategic concept;
    \item The selection of weapon systems and their combination;
    \item The quantitative and qualitative strength of the battle forces;
    \item The size of the reserves and their degree of invulnerability;
    \item The choice of the time and manipulation of the circumstances including deception;
    \item The exploitation of geography such as bases, areas of access, and approach routes;
    \item The formation of alliances, including secret prewar alliances of the utilization of allied territory to launch an attack from an entirely unexpected direction;
    \item The proper choice of a center of gravity of the operation; and
    \item The mounting of diversions, so that the opponent divides his forces.
\end{itemize}
The major problem is developing techniques to achieve technological surprise. If we assume that the enemy intelligence service watches the development of a weapon system from its early scientific inception to its use by operational forces, deceptive moves we make at any step in the process contribute to the ultimate surprise. For example, in the scientific field we can misinform and disinform to fool the opponent. Scientific misinformation would not be propagated in the form of false formulas which would not survive the first test, but it can be created by cryptic hints about programs and alleged results. Disinformation makes the enemy doubt the accuracy of his findings.

In addition there is secrecy. A classic method of achieving a technological surprise is secretly using foreign know-how. Another widely used method has been making an unobserved modification in a technologically inferior weapon system to give it a massive improvement in performance.

In the period of weapon development, surprise can be achieved through hiding and concealment, by pretended inadvertent showing of weapons and weapon components, by phony orders placed abroad for spares or scarce materials, and through a whole host of such stratagems that are not complex but must be planned into the production cycle.

One of the most effective methods is to start the development of several competing weapons, select one, and then give a great deal of publicity to the weapons that have been rejected and will not go into production. This was used by the Soviets when they exhibited the TU-31, equivalent to our B-36; the TU-31 did not go to production. In addition, rejected test models can be exhibited in operations in such a way that the enemy will be sure to see them and draw erroneous conclusions, while tests of the chosen models are concealed. If this is impossible, erroneous information can be fed into the technical intelligence stream and various red herrings can be used. In brief, the true testing operation can be enveloped in a lot of phony operations.

Another is to develop a weapon system to meet a specific operational requirement, then adapt it for a different operational employment. The Soviet MiG-25 is an example. Developed to counter threats never deployed, the original design was never taken past the prototype stage; it is now used for reconnaissance.

Similar tricks are available to hide production. The weapon system perhaps cannot be hidden, but there are many methods to make it difficult to obtain accurate performance data. As time goes on, several modifications that change the over-all characteristic of the weapon system can be concealed.

Errors contributing to surprise can be induced about the state of training and the precise deployment. In ground war, the effective concealment of a center of gravity is half the battle won. Generally, it is not correct to assume that military forces act consistently. Some nations tend to bluff; the German pre-World War I general staff operated on the principle that one should be considerably stronger than one appears to be. With respect to technological strategy, it is much better to create simultaneously impressions of greater, as well as lesser, capabilities.

\section{The Basic Purpose of Surprise}
The purpose of such maneuvers is to generate uncertainty in the mind of the opponent. Surprise may result from technology, but the actual surprise is not in the weapon system; it is in the mind of the commander and staff that surprise really takes place. Military commanders, not weapons systems, are surprised.

It’s probably worth repeating that: Surprise is an event that takes place in the mind of an enemy commander.

The devastating effect of surprise in the past has been caused by the fact that particular commanders and staff have for years conditioned their thinking according to firm expectations of enemy behavior and have carried out all their calculations within that framework. Suddenly, the basic assumptions are proved false by an unfolding operation. The result is a paralysis of thinking which often makes it impossible to carry out even those adaptations which could be accomplished within the time available.

There are a number of rationalizations that facilitate the surprise. For example, the assumption is frequently made that the enemy wouldn't do what we don't do -- "Why should he do that?" Another widespread notion is that the enemy would not do what he apparently is doing because, according to his opponent's calculations of the cost-effectiveness of a weapon system, there are cheaper and better ways to achieve the desired result. There are also such common beliefs as that the enemy would not pursue a certain course of action because he would duplicate a strength he already possesses, because he could not afford the expenditures involved, or because he would not be so dastardly.

By contrast, sometimes the enemy makes a spectacular demonstration or diversion for no other reason than to create attention and misdirect the estimator's interest. Then, after losing years in trying to figure out what the military significance of the stunt really was, the estimator arrives at the wrong conclusions.

In a discussion of surprise in a very broad sense, it is often overlooked that surprise about many smaller items has occasionally been truly decisive. If it is true that a major weapon system cannot be hidden, it also remains true that specific performance data can be manipulated in such a way that the enemy makes small errors. These errors may be within the margins usually allowed by statisticians, let us say 5\%. In actuality, speed differentials of 10 or 20, let alone 50, miles per hour may spell the difference between victory and defeat in combat. Similarly, such small differentials in, let us say, a radar performance, reliability of communications, or accuracy of missiles can be of the greatest significance.

In missile warfare, the reliability of the birds is crucial. If reliability is 10\% higher or lower than estimated, the enemy's strike capability is quite different from what it has been calculated to be. In addition, this reliability must be figured in the time dimension. Reliability can be very high if there are hours to get ready for the launch. If there are only 30 minutes, and if the force must be launched as the attack commences, the figure would change substantially.

When Minuteman II was deployed the reliability of its guidance and control system was about one-sixth of requirement. It took three years to overcome the difficulty, but then performance exceeded specifications. If the Soviets had attacked during this period, we would have been in a fine mess. Since the mishap was widely rumored, the Soviets probably knew about it -- fortunately, the U.S.S.R. lacked adequate strength.

\section{Historical Examples}
In 1937, the Germans won an air race in a spectacular manner by stripping down their Messerschmitts while the other nations entered fully-equipped fighters. Presumably the staffs understood this particular trick, but the public, the reporters, and the political decision makers were fooled. This, of course, is an example of combined technological and psychological strategy.

The most intriguing aspect of the history of aerospace war and the role of the surprise is that very professional staffs have been deceived about the most basic elements of this new type of war. At times this has been self-deception; at other times they were deceived through deliberate campaigns.

There was once the notion that the airplane was not really a militarily useful weapon. When this notion was dispelled -- it took years -- it was believed that the airplane would serve its purposes best by direct support of the ground battle. Consequently, the range of the aircraft was considered to be of no importance and it was thought that the range should rather be short. Later there was a great deal of doubt about the proper targets for strategic bombardment. The effectiveness of strategic air war was a matter of considerable dispute, largely because the interrelationships between industry, battle strength, and time factors involved were not understood. Furthermore, some air warriors overlooked the recuperation factor.

Similarly, during World War II there was a debate about whether the air weapon should be used for only one purpose -- against industrial targets. After World War II, similar arguments raged with respect to nuclear weapons, jet aircraft, long-range bombardment versus forward bases (the question was ill-conceived as an either-or proposition), and, of course, space and air bombardment in Vietnam. Few debaters ever look at the whole range of arguments, and non sequiturs usually abound because emotions become involved in the arguments.

Another frequent source of error is that the versatility of the weapon system is underrated. The aircraft obviously is an excellent purveyor of firepower. But often ignored are its uses for demonstration, reconnaissance, the transport of goods and troops, command posts, and damage assessment and its possible employment in big as well as small wars. Some people who know such capabilities only too well, but for political reasons don't want new equipment, put up smoke-screen arguments against it.

The Strategic Defense Initiatives debates are similar. By an odd coincidence, all those who oppose SDI think it will not work. We do not recall one scientist of note who would like to see it deployed but believes it is just too expensive, or too difficult. The result is that what appears to be a technological debate is in fact a political one; but the fact remains that strategic defense offers one of the most decisive opportunities for strategic surprise in all history.

\section{Breakthroughs}
The many facets of developing, acquiring, and operating advanced weapons systems illustrate the need to consider surprise as one of the key elements of technological strategy. Technological warfare includes the anticipated breakthrough, but the breakthrough need not be a surprise.

In fact, it could well be tactical to announce a happy breakthrough that for a while cannot be countered by the enemy. His inability may come from one of two sources -- technological inferiority or inferiority in the decision-making process. Naturally, the combination of these two deficiencies would increase the lead of the opposing power. In the end, unless he is defeated, the opponent would catch up with the new technique. The strategic impact of the breakthrough is a function of the duration of the one-sided advantage.

While surprise has its advantages as far as modernization of the force in being is concerned, the breakthrough has the potential of pushing the state of military art into an entirely new field that may lead to clear dominance. This is the role space warfare will play in the future. At present after three decades of space efforts we face an unprecedented situation: a clear military superiority in space potentially can ensure denial of creating a countercapability. There may be a significant novel feature, namely, that even without war such denial could be long-term.

The ability to deny an enemy access to space is essentially the ability to deny him world power status. You cannot be a global power without access to space.

\section{Exploitation of Surprise}
Initiation of war usually is the object of a great deal of surprise planning. Prior to the initiation of war, the planning of the opponent can be rendered ineffective by such techniques as misinformation (the propagation of misleading and false knowledge) and disinformation (the propagation of news designed to induce the enemy to disbelieve existing truthful and reliable information and buy false new information instead). The aggressor can use the time-honored techniques of single and double deception\footnote{Double deception is best explained by the story of the two Jews who met on a train in Russia. Aaron asked Moses, "Where are you going?" Moses answered "To Pinsk." Aaron replied, "You say you are going to Pinsk so that I will believe you are actually going to Misnk, but I happen to know you really are going to Pinsk. So why do you lie?" 
In military parlance, if A plans an operation he would not try to hide his plan, but would make sure that B assumes this particular plan is being advertised because it will not be implemented. The German deception plan of 1941 that preceded the attack on the Soviet Union was planned as a single deception but actually worked as a double deception.} to cloak the steps leading to his attack and induce the opponent to misread his intentions.

To meet deceptions of this sort, the strategic planner by necessity must plan against a war that might come regardless of the probability that it will not. This planning must be based on the enemy's capabilities to strike rather than on his professed intentions. The fact too often ignored is that intentions can change very rapidly, and that implementation of the new intention might require a shorter lead time than improvisation of defense against an attack that was not expected.

Under conditions of nuclear war, the importance of deception techniques is growing ever more rapidly. Arms Control negotiations must necessarily be a part of an aggressive strategy under modern conditions; the aggressor must use deception techniques to bring about disarmament arrangements which reduce the size of hostile forces in being and thus greatly simplify his planning. For example, the reasonableness the Soviets seemingly displayed in the initial SALT talks may denote (a) a turn toward peace, (b) a maneuver to delay U.S. reaction to the missile build-up in the U.S.S.R., and (c) an attempt to gain a safe rear and increase supplies for a Soviet attack on China.

The above was written in 1969. As we look back now we see that the second premise was correct, with the result that the Soviets gained a clear advantage in ICBM numbers and performance, and in military exploitation of space. (The Soviet Union ha a number of 100-kilowatt powered satellite radars in orbit; the US has yet to put up a 10 KW radar.)

Surprise can be achieved through disarmament and arms control arrangements and the use of propaganda and diplomacy, on one hand, and through counterintelligence, introduction of misleading intelligence, and infiltration into intelligence and policy-making staffs, on the other. As an example, before they had completed operational tests of their antimissile system, the Soviets refused to discuss an atmospheric test ban; afterward they rushed to agree before we tested our weapons concept. Other surprise techniques which may be applied could involve the holding of deceptive maneuvers, the building of dummy forces and targets to divert firepower, the employment of electronic equipments that would not be used in war, and electronic deception on a broad scale.

One important technique of surprise of which American writers seem to remain unaware, is provocation\footnote{The Six-Day War in the Middle East has made the concept better known.} This word in English usage denotes the provoking of an opponent into a rash act, but in the Communist dictionary it also means entrapment and instigation of a fight between third parties. Many wars have been started by provocations deliberately engineered by the aggressor; the purpose has frequently been to shift the onus of aggression from the aggressor to the defender.\footnote{As the Russians say, "If I attack you and you don't defend, there will be no war; if I attack you and you defend yourself, there will be war and you caused it."} 

Other purposes may be to force the defender to make some sort of premature move and thus expose his strategy, or to get him embroiled in a struggle on another front so that he would disperse his forces and lose control. Such an effect could be achieved, for example, by forcing the defender into a limited war in a peripheral theater and gradually cause him to invest ever-greater military strength from his forces in being into this limited operation. Thus, he would expose his main base to effective attack. If he can be induced to use obsolete equipment in the diversionary war, the victim may never develop the kind of weapons that will be used in the decisive combat.

So-called preemptive strikes also may play a great role in surprise. The attacker could proceed by a combination of double deception and provocation to make open preparation for attack and to evacuate his cities. Then by other surprise techniques he could divert the defender's fire to false targets and achieve military superiority. Certainly moves of this sort are extremely risky, because the defender has surprise options of his own and may see through the deception. The risk can be reduced through a first-rate intelligence system, a superb early warning system such as would be provided by deploying even the most elementary Strategic Defense System, and good penetration of the opponent's military apparatus and inner decision-making cycle.

Strategic planning aims at the exploitation of weaknesses and vulnerabilities, just as the wrestler tries to apply holds that force his opponent to submit. But the strategist has one advantage over the wrestler: he can contribute to the creation of vulnerabilities in the opposing force.

Creating vulnerabilities is an area where the problems of force and budgetary levels become highly significant. They can be created by an opponent who uses political means to achieve surprise. With low budgets there will always be a great tendency to cut corners, and that means that many of the support systems needed to operate weapon systems effectively will be eliminated or reduced to insufficient numbers. Very often it becomes a question of whether it is more advisable to buy firepower and delivery weapons than to harden the missiles or acquire such items as warning systems. Sometimes the choice is between offensive and defensive weapon systems.

If the aggressor can, through the employment of political means, manipulate budgetary and force levels of intended victims in a downward direction, the effectiveness of the opposing forces will be greatly reduced. Fundamentally, with a low budget it is very difficult to maintain alternative weapon systems simultaneously, and even more difficult to maintain forces based on different technologies. It is extremely difficult to provide them with good warning and protective features, to acquire suitable shelters for population and industry, and to bring new systems into being. Consequently, low defense budgets and low force levels aid the attacker in his strategic planning by reducing the complexities of his operations. Political operations in both the economic and diplomatic fields may be used to reinforce the natural tendency of the defender to save money on defense. These operations will have as their twin goals the reduction of strategic complexities through the lowering of the defender's budgets and the achievement of a state of relaxation in the victim. Then, when the attack comes, on the victim's allies and/or on his homeland, he will be unable to believe it has happened and be unprepared to defend himself. In this case, the last phase of the battle may not be a sneak attack at all; the defender may know it is coming and be unable to do anything about it.

To repeat: surprise techniques are available to both the attacker and the defender. Because we are firmly committed to a defensive strategy it is vital that we prevent surprise. We must understand also that capabilities for surprise exist for us and we must emphasize such capabilities.

These come directly from the basis for surprise: uncertainty. Although the attacker has freedom in choosing his surprise moves, the defender can do a great deal to increase the uncertainty in the mind of the attacker. If the attacker has no uncertainty about the enemy, it is child's play to plan operations that can be decisive. If instead he experiences a great deal of uncertainty, even the planning of surprise operations becomes extremely difficult.

For example, a major purpose of strategic defense is to create uncertainties. If the defender does not have this capability the attacker will be certain that he has a completely free ride. If the defender has active missile defenses and the attacker is in doubt about whether its effectiveness lies between 50 and 90 percent, the attacker's strategic plan is greatly complicated. Suppose he assumes it is 50 percent, but it is actually 90 percent effective. Then he will fail in attack. Suppose he assumes it is 90 percent but it is actually 50 percent. In this instance he may not attack at all. Suppose his experts argue about whether it is 60 or 85 percent. In this case, the decision makers' will may be weakened. By manipulating the attacker's understanding of this situation, the defender may achieve considerable advantages.

The interplay between achieving and preventing surprise is one of the decisive elements of modern war. Speed appears to give the attacker greatly enhanced possibilities of surprise, but the defender is not without his options as well. The key to being the surpriser or the surprised is initiative, which in turn is based upon planning.

\section{Conclusion}
In guarding against technical surprise, it is important to keep its effects in the proper perspective. Technical advances generally and technical surprise in particular are steps to more decisive measures. Technology makes possible tactical, strategic, and timing surprise, and also provides systems for preventing surprise. It contributes to strategic deception, or prevents it. Surprise and deception are most vital when they contribute to or maximize the effectiveness of modern weapons. If our technological advantages are not exploited, while those of the U.S.S.R. are, we will inevitably lose the Technological War. Put differently: we must not be surprised about the fact that this is a Technological War and we must never be deceived about our relative technological standing.

Success in an operational approach based on deception and surprise depends on total orchestration of the types of conflict, not on the effectiveness of each element. Partial successes attained and exploited in many areas will offset the failures that will occur in others. The net result is that overall success is rendered more likely.

If the defender understands this particular aspect of the problem, he can devise many actions through which aggressive stratagems are neutralized. He can maintain force levels, both quantitative and qualitative, that preclude a successful attack. The defender must move constantly during the period of so-called peace, to keep abreast of technical and strategic developments. He must initiate actions to which the attacker must react, using resources that would otherwise be employed against the defender, and must initiate these actions in time to prevent the aggressor from achieving a significant advantage. Success in this game will mean that aggression by nuclear weapons would be unthinkable, simply because the aggressor would remain confined to an incalculable but low probability of success.

The really important point is that war has not become unthinkable simply because weapons of mass destruction have been invented. The prevention of war is just as much a strategic undertaking as preparation for aggression. If the strategy of prevention is effective, the aggressor will be blocked. If, on the other hand, it consists merely of dependence on passive deterrence and on weak retaliation, the strategy of prevention is doomed to failure.

For the Communists, surprise is vital to successful aggression. For our part, through the application of a rehumanized strategy surprise can be our path to the initiatives and maneuvers that suppress aggression.

\textbf{The only thing that is worse than being taken entirely be surprise is to be taken by surprise after repeated warnings that one is going to be taken by surprise. The former is shocking. The latter is devastating.}
