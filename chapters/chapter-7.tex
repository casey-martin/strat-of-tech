\chapter{The Nuclear Technology Race}

\section{Foreward}
\subsection{1997}
Revised from Kane notes up to the point marked. This chapter is accompanied by another on the subject by Dr. Kane.

Of all the technologies since World War II, the one which epitomizes a strategy of technology is the silent war in developing nuclear technology. The U.S. pursuit and application of this postwar breakthrough is science and development has followed two paths.

In one the U.S. excelled and still excels; in the other the U.S. consistently demonstrated its failure to apply its innovative skills to national strategy. In this chapter we deal with the technology first, then relate it to the issue of strategy.

Throughout the period there are two major themes: fear of nuclear technology, and the development of weapons for deterrence.

The reasons for fearing nuclear technology are obvious. Nuclear weapons have sufficient power to destroy a great part of the Earth's population and wealth in a short time.

It is also well to remember why the U.S. places so much emphasis on nuclear technology for deterrence. In the late 1940's and early 1950's it became clear that the U.S. had no choice but to erect a defensive perimeter to assure its freedom and that of its allies in Europe and Asia. Major studies were conducted to examine two alternatives: dependence on conventional technology (a study that lasted three years), or dependence on nuclear weapons.

The cost tradeoffs clearly favored dependence on the nuclear deterrent. Matching the Soviets in size of forces while depending on a semi-mobilized economy was clearly not acceptable because of both the cost and political factors: the U.S. population was unlikely to endure more years of mobilization.

The U.S. began research, development, and acquisition of nuclear weapons to deter central war and for battlefield deployment to deter war in Europe and Asia. However, though these actions were deemed essential, the rate and timing were episodic, determined more by Soviet initiatives than by deliberate U.S. planned application to a long-term strategy.

Strategy reflects a struggle between decision centers.

The U.S. goal has been to preserve the status quo through a defensive strategy which is based on offensive forces. These forces and the nation itself have been (and in 1989 remain) essentially undefended. Meanwhile the Soviets continually tried to preserve the initiative and freedom of action. Their strategy has two aspects: a dynamic technological effort to try to match the episodic advances of the U.S. and a diplomatic/propaganda effort to constrain and delay U.S. technology by confusing the U.S. decision process, generally by invoking fears of global annihilation.

\subsection{1988}
There have been many advances in nuclear technology since this chapter was written. Given the dynamism of the field it would be surprising if there had not been.

In 1969 our analysis of nuclear technology focussed on lost opportunities. We did not have a broad strategy for exploiting all the potential applications of nuclear technology.

We did, however, have a nuclear strategy. It has been in operation since somewhat before the first edition, and continues to this day. The strategy is very narrow; but very, very successful. The objective has been continuously to improve our weapons in spite of all constraints. The weapon technology goes hand in hand with the inertial guidance technology. As accuracy has gotten better and better, yield has gone down, and weapon effects have gone up. Furthermore, our weapons are longer-lived, have become more reliable, and have been very economical in the use of critical nuclear material.

As we have decreased yield without giving up military (and even improving) military effectiveness we have reduced the amount of critical material in each weapon. Thus, we are able to "mine" obsolete weapons for their nuclear material and re-use it for newer, more efficient designs.

That strategy, narrow as it is, can only be called an unqualified success.

\section{The Applications Effort}
In the fifty years since Nils Bohr announced the splitting of the atom, nuclear technology has grown and matured -- and become the most controversial technology in history. As we approach the end of the century, the issue is whether or not nuclear technology will continue to be constrained from its full potential. In the immediate post-war period several landmark studies (as for example the Lexington Report) identified applications to an array of military and civil applications (See Chart 17). Most of them were explored. But at the same time there was a raging discussion of ways to limit those applications or to "put the nuclear genie" back in the bottle. That situation still prevails at the start of the last decade of this century -- new applications are being invented; new attempts are being made to prevent them.

The list of military applications explored covers nearly the entire range of propulsion and weapon systems.

In the initial period the focus was on nuclear weapons design and production. Two major designs were pursued: Implosion and insertion. The objectives in weapon design were efficiency and safety. As for efficiency, there were two major objectives: improving the yield to weight ratio and decreasing the amount of critical material used. Safety aspects concentrated on the bombs themselves, including extension of life of the weapons, and maintenance of reliability. During this period also, an entirely new weapon was designed -- the hydrogen or "H" bomb.

Principally under the influence of the ICBM program, design shifted from weapon development and production to that of weapon systems. The marriage of a small weapon with a rocket booster led the way to an integrated approach. The first major product was the MIRV'ed ICBM, but others were found in the SLBM, field artillery, and tactical fighter delivered weapons.

Beginning at about the same time, other technologies, principally electronics, began to play a major role in nuclear weapons. The internal guidance system of the Post-Boost Vehicle which carries the MIRV's greatly improved the accuracy of weapon delivery. That technological innovation meant that the yield of the individual weapon could be reduced while still maintaining weapon effectiveness as measured by SSPK (Single-Shot Probability of Kill).\footnote{
Accuracy is much more important than yield in increasing the SSPK. For example [[ give examples from Rand wheel ]]} 

Another important application of electronics came from the political requirement for absolute control of each weapon. The concern was that aircraft-carried weapons could be employed by the aircraft crew on their command. Thus, inadvertent, accidental, or deliberate but unauthorized release could occur and nuclear war could result. Consequently, Permissive Action Links (PAL's) were designed and installed on nuclear weapons. For ICBM's in their silo's, a "turn-key" system was installed so that no one crew member could launch a missile, because two members would have to "turn their keys" in a prescribed sequence and on receipt of a coded message in order for ICBM launch to occur.

Implementation of the INF will remove the newest, most effective nuclear weapons from the U.S. stockpile. This is a reversal of prior treaties which resulted in or permitted removal of older, less efficient weapons while retaining the most modern ones. One of the effects of the INF is thus to increase the average age of the U.S. nuclear weapons stockpile.

In the decade of the 1990's, nuclear weapon technology will see a new phase -- transformation to "wizard" weapons. During the war in Viet Nam advanced guidance technology, notably lasers, was adapted to World War II conventional bombs to make them more effective. Heroic feats of air delivery against selected elements of the power plants in Hanoi with a CEP of 14 feet were achieved with "smart bombs". These early highly accurate guidance systems will be adapted to nuclear weapons to produce in effect zero CEP weapons, meaning that no passive system of silo hardening can assure the survival of second-strike weapons.

Optical guidance, map matching, radar guidance including laser radar are available. The weight of such guidance systems will be measured in ounces, not pounds; their mass will be practically zero also. These application to nuclear weapon design with still greater improvements in yield to weight ratios, will result in new weapon capabilities. Such advances will cascade into small, more effective weapon systems.

Given the advances made in the 1960's in discriminate nuclear weapons, the ultimate will be highly effective performance with controlled energy release from small weapon systems.

On the other hand, nuclear weapon technology can be applied to ballistic missile defense by X-Ray lasers which can have destructive effects at very long distances in space. Whether based in space or on the ground with relay mirrors lasers can achieve such effects nearly instantaneously. Such lasers and other "speed of light" weapons will be possible in the next century if not well before.

But as has been the history of nuclear technology, the development of "wizard" nuclear weapons and "speed of light" weapons will be constrained and perhaps prevented by policy decisions. And those decisions will flow from the continuing fear of the atom, fear which has retarded many potential applications.

A key factor in the evaluation of policy will continue to be the Soviet drive for a total ban on testing. A comprehensive test ban would mean the end of nuclear technology.

Nuclear bombs grew from the 20 kiloton weapon of 1945 to the 60 megaton bomb exploded by the Soviets in 1961 and 1962, when they abrogated the "gentlemen's agreement" not to test nuclear weapons in the Earth's atmosphere. As nuclear technology matured, the explosive power in bombs declined from the multi-megaton range to that of the low kiloton. Such bombs are carried by fighter and bomber aircraft, ballistic missiles (ICBM, SRAM's, and SLBM), and cruise missiles.

Nuclear artillery rounds were developed, deployed and modernized for battlefield operation, particularly as part of the U.S. deterrent to Soviet attack on NATO.

Nuclear air defense weapons were deployed in Europe and the U.S. Nuclear weapons for ballistic missile defense and ASATs were deployed. Nuclear depth charges were designed. Cassaba and Howitzer were designed as nuclear ballistic missile defenses.

In propulsion technology, nuclear powered engines were developed for the Camel long-range bomber; nuclear powered cruise missile as well as SLAM, the nuclear ramjet; NERVA, a nuclear space propulsion system; nuclear reactors for ships, both surface and submarine; nuclear propulsion for spacecraft (SNAP) was used; and nuclear power systems for space stations and lunar bases were designed. A design to propel large satellites and spacecraft by a chain of nuclear explosions (Orion) was developed but never implemented.\footnote{
ORION could have put a two million pound payload on the Moon in one flight. This is more than sufficient for a complete Lunar Base.}
 

The Soviets for their part have developed and orbited many nuclear reactors. Their 100 kw Radar Ocean Surveillance system is the largest power plant put in orbit by anyone. (The largest US system has less than 10 kw of power.)

On the commercial side, nuclear power for the generation of electricity became the mainstay of France and other countries. Nuclear explosions for peaceful purposes such as building canals were actively considered by both the U.S. and U.S.S.R. for a decade and dropped.\footnote{
The Soviets seriously considered very large geographical engineering projects such as diversion of north-flowing rivers and the creation of an inland sea.
} 

All these applications were constrained by fear; fear of accident, pollution and unknown effects, and the overriding dread that the use of even one nuclear weapon (even by accident) would lead to the end of mankind. To allay fears, emphasis was placed on safeguards and constraints. Nuclear weapons on aircraft, for example, were controlled by Permissive Action Links (PAL) so that they could be used only on authority of responsible civilians. Most of the applications for propulsion were dropped because of the impossibility of safe military operations. The nuclear-powered aircraft was to have been flown in remote areas of Utah and a special hangar was built for it even though the program never survived the design stage. A nuclear reactor was flown on a B-36H but it was not used to power the airplane. Of those propulsion applications only the nuclear reactor to power submarines survived and matured.

In like manner, nuclear "plowshares" never became a real program. Nuclear power for generation of electricity survived, albeit controversy surrounded individual plants and caused delays in construction, cancellation of programs, and even abandonment of plants.

Very much related is the issue of disposing of nuclear waste materials. The search for suitable sites has dragged on for years, hindered by concerns for pollution of the water supply and other health hazards.

Nevertheless, invention continues. New ways to focus energy produced by nuclear explosions were developed. One application was postulated for the X-Ray as a source of power to destroy enemy ballistic missiles. Tailored weapon effects for discriminant employment were developed for the "Safeguard" ABM program and battlefield weapons, and the most controversial of the inventions was the "neutron bomb" which could kill enemy forces by enhanced radiation with little collateral damage to structures and the environment.

In parallel with the technical effort was the strategic struggle with the U.S.S.R. which continues to this day.

\section{The Basic and Continuing Role: Deterring War}
Throughout the early decades of the struggle the U.S. developed and maintained adequate power to deter. Soviet advances and diplomatic and political maneuvers did not give them a decisive advantage. The U.S. episodic development, painful as it was politically maintained an adequate posture.

The development of nuclear weapons undoubtedly spared Europe from another frightful war of the dimensions of World War II. Had it not been for the American nuclear monopoly, it is highly unlikely that the USSR would have ceased expansion with East Germany, Czechoslovakia, Poland, Hungary, Rumania, the Baltic Republics, etc., in 1945-1948. The United States retained little conventional ability to stop the Soviets short of the Pyrenees, if there, and except for the threat of nuclear weapons, Europe would have fallen and would have pulled America down. As a French expert puts it, "The disappearance of nuclear deterrence would be a frightful catastrophe, for we should then lose the benefit of the stability created by the atom in our rapidly evolving world. Actually, if the United States and the United Kingdom had not developed the new weapon the Nazis would have done so, and thereby would have won World War II. The chimneys of the extermination camps would still be smoking.

If they are owned by both sides, nuclear weapons create a nearly unique historical situation, one in which the loser of a war may still retain sufficient striking power to badly damage or destroy the winner. This may be negated by defensive systems: but no matter how good the defense, the aggressor cannot rely on them for 100\% protection. Some of the enemy's weapons may get through the defenses no matter how badly the enemy has been hurt.

Thus, because of nuclear weapons, deterrence becomes possible and the defensive grand strategy of the United States could be effective and may be ultimately successful. Without such weapons, we would be required to retain ground, sea, and air forces of enormous size to deter Communist aggression and we would have to deploy them overseas. Nuclear weapons create a strategic environment in which deterrence is at least theoretically feasible in terms of being economically bearable and of preventing the use of all-out war to settle a conflict.

However, the enormous power of these weapons creates still another strategic possibility: a decisive military advantage could be gained through victory in the Technological War, i.e. through technological competition without violence.

To some extent, qualitative inferiority can be compensated for by sheer numbers of weapons. It is true that the more deliverable nuclear weapons the defensive side retains in its inventory, the less attractive the situation is for the attacker; but, we must repeat, modern technology is very fluid. Truly revolutionary advances in the field of modern weapons--advances that would upset all existing military relationships--are not only possible but, if one side does nothing while the other actively seeks to exploit new technological potentialities, such upsets are well-nigh inevitable. There is no standing still, and no going back. In a world of conflict and dynamic science, the only rational policy is to pursue the Technological War diligently.

\section{The Initiative}
Nuclear technology also presents a splendid opportunity for seizing the initiative in the conflict. The United States has a defensive grand strategy. By active development of nuclear technology, we can force the U.S.S.R. to invest heavily in weapons to counter our advantages, to defend itself against our new weapons, and in general to use resources in ways that cannot harm us. Most of the advanced theoretical work in nuclear physics takes place in the West. Most of the facilities for active development of useful nuclear engineering devices are in the United States. We require only a rational strategy of nuclear technology--and the will--to take advantage of our superior facilities and resources. If we embarked upon such a course we would do more to cool down the Cold War than by almost any other technique.

\section{The Shape of Things To Come: The Baruch Plan}
The opening round of the struggle came when the U.S. presented the Baruch Plan, under which the U.S. offered to share its nuclear technology with the Soviets. That offer was summarily rejected by Stalin, who declared the U.S. to be the enemy and pushed ahead with his nuclear technology program with emphasis on nuclear weapons. Surprising the U.S. and the world, the Soviets exploited the programs they had started during World War II (aided by espionage of US technologies), to explode their first atomic bomb in 1949.

Within the U.S. a bitter debate arose over whether or not to pursue the next phase of weapon development: the hydrogen bomb. This well-known struggle was settled with the U.S. decision of proceed but the Soviets had the first bomb. The U.S. delay resulted from the gamut of assertions that would be repeated at each subsequent decision point:
\begin{enumerate}
    \item If we don't, the Soviets won't.
    \item Don't trigger a new round in the arms race.
    \item Let's negotiate a ban on the new development; better, let's negotiate nuclear disarmament.
    \item We're heading for annihilation of life on Earth.
\end{enumerate}

\section{The Second Ploy: The Test Ban}
In 1945, the United States enjoyed absolute superiority in nuclear weapons technology. Twenty-five years later we are not certain that we are ahead in any area of nuclear weapons research, and we know that we are behind in some. Yet, during that time we have invested far more resources in nuclear weapons research and development than the USSR Despite such massive investments, our lack of a strategy and strategic sense has allowed the enemy to close the nuclear gap, and even to create one in his favor.

The first phase of this battle in the Technological War was dominated by our failure to engage in technological pursuit. We did very little to exploit our nuclear monopoly, and nothing to hinder the Soviet's development of weapons of their own. Indeed, through our lend-lease policy we sent important nuclear resources, including enriched uranium, to the USSR, greatly aiding their weapons development. Espionage also played an important role in this phase of the battle and reduced the Soviet lag time substantially. But our deliberate decision not to engage actively in development of new nuclear weapon systems was our crucial failure. Given Soviet industrial resources, all the espionage in the world would not have enabled them to catch up with us if we had been running at only one-half our potential top speed.

The second phase of the battle was the race for the thermonuclear or hydrogen weapon. This story has been told often enough and will not be repeated here, but the key decision was to allow technologists who for political reasons wanted the hydrogen weapon to be impractical to govern the allocation of resources and thereby to starve the H-program. Not surprisingly, men who believed the new weapon to be impossible, who devoted few resources to its development, and who wished it never could be built, were unable to construct the device; yet, as it happened, at least three feasible approaches to hydrogen weapon construction were discovered, none particularly difficult or complex. In this phase, the Soviets almost outstripped us, and did develop a bomb before we did.

\section{The Test-ban Strategy}
An even more decisive phase of the nuclear development battle came in the late 1950's, and illustrates the highly successful orchestration of technological and nontechnical resources into a strategy for the Technological War. The fact that this successful strategy was conceived by us, developed by the U.S.S.R., and employed against the free world should not prevent us from profiting by its example. The test-ban phase of the battle was protracted over several years, and during the entire time the initiative lay with the U.S.S.R. on the one hand, and American demagogues and professional disarmers uber alles on the other.

The first salvo of this battle came with exploitation of the sincere concern of U.S. scientists and conservationists over the possibility of atmospheric contamination due to fallout from tests, coupled with the hope that further developments in nuclear technology would never be made. It was this fear of contamination of the atmosphere that led to the atmospheric and space test bans, and it should be recalled when we examine the consequent course of the battle.

At first the U.S.S.R. and the disarmers skillfully manipulated sentiment for banning nuclear testing, through public statements in favor of a test ban accompanied by impossible conditions for a test-ban treaty. During this time, the U.S.S.R. rapidly conducted tests, then, before analyzing the test data, proposed a gentlemen's agreement or moratorium on testing. The United States concurred at once and testing ceased.

However, while we congratulated ourselves, the Soviets prepared for a new test series which took place, in violation of the moratorium, after the data of the previous tests had been evaluated and planning of the new tests had been adjusted to these findings. Test shot after test shot, all designed to increase knowledge of large-yield weapons and high-altitude explosions, was detonated with monotonous regularity, while the United States raced to respond with a new test series of its own.

In designing our own series, however, we had no real strategy; thus we conducted tests for many different purposes, and completed no series before we were again caught in the test-ban trap. Before that time, however, the Soviets, directed by their own strategic analysis, tested first large-yield weapons, then weapons for defense against ballistic missiles. They launched rockets from their center at Kapustin Yar, near the Volga, and shot them down with interceptors launched from their defense testing complex at Sary Sagan, near Lake Balkhash. They fired ICBMs with nuclear warheads, allowing the weapons to detonate after reentry. They exploded defensive warheads directly under their scientific satellites. Finally, when their preplanned series was complete, they changed their diplomatic position.

Instead of a complete test ban without inspection, they now insisted on a partial test ban to include not only atmospheric tests but also those in space. Note that there is no public health reason to include a space test ban. This was quickly accepted by the United States in the Treaty of Moscow, and we then found ourselves in a new situation. Under the terms of this treaty, the U.S.S.R. was free to conduct underground tests of small weapons in which the United States was in the lead; but the United States could not conduct, either in the atmosphere or in outer space, tests of the really large yield weapons or defensive weapons in which the U.S.S.R. was in the lead. This situation has continued to this day.

Note particularly that the reason for the original cry for the test ban was that tests threatened to contaminate the atmosphere; note also that this cry was raised because of our emotional dislike of nuclear weapons and our desire to ban all tests; the final result was a ban on tests not only in the atmosphere, where health considerations are important and a test ban is desirable, but also in space, where no atmospheric contamination is possible.

Because the Soviets had a strategy for conducting the nuclear development battle of the Technological War, they were able to maneuver us into a position of temporary disadvantage and deliver us a set-back, i.e. they imposed upon us unilateral military handicaps and they demonstrated the psychological manipulability of the United States. The Soviets then turned to technological pursuit, deploying missile defenses making use of technology that we do not have and can get only with difficulty if at all and testing small-yield weapons underground to close that lead which we had held.

\section{Another Strategic Failure}
The neutron weapon could be an important element in the next phase of the struggle. Its significance lies beyond arguments about feasibility; and indeed, the only arguments about funding research in the vital area have been arguments about feasibility. The few strategic points made by opponents of neutron technology have been erroneous.

Most of the discussion has centered around technical difficulties. William Laurence, for example, quoted "scientific opinion" that "it is scientifically unlikely that anybody can perfect an N-bomb for nearly half a century." Sometimes such predictions may be right or false but in this case it was plainly silly; it was inspired by those advisors around Kennedy and McNamara who did not want the weapon in the first place. Their opinion was not, however, based on strategic need, which received no consideration, but on general opposition to nuclear research bolstered by the overkill thesis. In making a funding decision, however, strategic requirements should be the major consideration.

The neutron weapon produces an explosion with blast in the same order of magnitude as that of a large TNT detonation. It develops little heat and negligible long-term radioactivity and fallout. That is, the neutron bomb is a kind of "death-ray" which destroys organic tissue, has great power of penetration, and does little damage to property.

There are obvious military advantages to such a weapon. Even if they cannot be constructed at low cost and air-deliverable weight, neutron weapons could be useful as atomic land mines to impede the advance of hostile field armies. In fact, the use of neutron weapons to halt enemy invasion of U.S. allies is their most obvious application; such weapons are not subject to the same objections as are other tactical nuclear weapons. They create no fallout, and destroy no large areas.

The real importance of the neutron weapon, however, is that it was a new and unprecedented nuclear technique which could lead to a revolution in weapons technology. It is not only what the N-bomb does that counts, it is also that research may ensure technological progress in the nuclear field. This is, to be sure, an intangible factor that is unpalatable to those who fear further progress.

The tactical utility should not be ignored, however. The addition of neutron weapons to the U.S. arsenal could be a major factor in future hostilities by allowing the United States to engage in war with small commitments of men and resources. By preserving its resources for the decisive Technological War, the United States would continue to function in the proper role of the arsenal of democracy.

The usefulness of neutron devices for small wars should be obvious. Enemy troops, not allied real estate, should be the targets of air weapons. One reason we have never been able to use nuclear weapons in small wars is that they leave behind them a swath of destruction and residual contamination, destroying the areas we hope to liberate. Neutron weapons do not suffer from these defects.

In a full, centralized war, the neutron weapon makes possible a new strategy. Concentration of neutron devices on the Kremlin and other known enemy command posts is preferable to blasting entire cities. If we have weapons of this kind we can use them to good psychological advantage. We could inform the peoples of the U.S.S.R., particularly such dissident minorities as Ukrainians, Estonians, Turks, etc., that we are using neutron weapons because we are not at war with the population of the U.S.S.R. but are compelled to eliminate their oppressors, who want to be oppressors of America also. If our strategy were designed to discriminate between friend and foe it would be in the self-interest of many citizens on the other side of the battle line to help us get rid of the real aggressor. With large indiscriminate weapons we will inevitably kill those who would be on our side.

There is an excellent chance that the neutron weapon will greatly improve antiaircraft and antimissile defense systems. The neutron device may be marginal for antiaircraft defense because the blast and fireball of existing nuclear weapons presumably would have a radius of destruction greater than the radius of neutron flux; on the other hand, the neutron weapons will be absolutely clean, and this is a great advantage. In space, radiation is the only long-range effect that can be obtained from any nuclear explosion. Whether neutrons or some other type of radiation such as X-rays (the kill mechanism of the first-generation ABM) are more suitable for the destruction of incoming warheads is unanswerable so long as the neutron device has not been tested. For that matter, combinations of radiation types may prove to be the best proof against the ICBM designer's skill. In any event, if radiation is the main nuclear agent for ICBM defense and the only practical kill mechanism in space, we can hardly neglect research in this field, particularly in radiation weapons such as the neutron device.

Scientists have opposed the neutron device because many of them are instinctively opposed to advancing nuclear techniques. Some of them have stated their opposition to the neutron weapon honestly in those terms. More frequently, however, it has been alleged that neutron weapons are not important because, one, they could not be built; or two, if built, they could not be produced cheaply or in practical configurations; or three, even if they could be incorporated into weapon systems, they would not add to our existing capabilities or do military jobs better than existing nuclear weapons.

Other scientists have stated that in their judgment the neutron weapon would be useful only for ground combat and have alleged that this development commands only a very low priority. The fact of the matter is that in the past scientists have made very bad strategic and tactical analyses because they often argued on a priori grounds and looked only at fragments or segments of the overall operational requirement.

As a result of this opposition, development of the neutron device has been delayed. This is the old familiar story of the vicious circle: if budget and priorities are established on an assumption that a particular development is not promising or useful, only mediocre results can be expected. Fortunately, science marches on.

We had an instructive experience with the hydrogen weapon, which was delayed because some scientists did not believe in the need for the United States to have this weapon. When, because of dramatic Soviet progress, the decision was made to go ahead on the program with full power, the feasibility of the H-bomb was still very much in doubt. However, once the program really got under way, the necessary solutions were speedily found.

The same happened with neutron weapons. The technical problems were quickly solved. There remain the usual arguments for constraining U.S. nuclear technology.

Fear has been expressed that by pushing neutron technology we will push the Soviets in the same direction and thereby disturb the stability of the strategic balance. This deserves some attention. First, the so-called stability of the strategic balance is an illusion. In time, every system in our strategic inventory will become obsolete. Second, the nature of nuclear weapons makes arms races in this modern era qualitatively different from those of the conventional weapons period. An increase in conventional weapons capability gives a power increased confidence in his ability to win a war without disastrous results. The same is not true of a mutual increase in nuclear capability. We will discuss this more fully in the final chapter, but it is well to keep in mind the conclusion of General Beaufre: "A conventional arms race produces instability, whereas a nuclear arms race produces stability."

Moreover, the U.S.S.R. is inevitably making nuclear progress; but the strategic situations of the Soviet Union and the United States are by no means parallel. We have no intention of invading Communist territory, but we want to prevent the Communist invasion of the free world. Consequently a battlefield weapon that minimizes civilian casualties is to our advantage. Otherwise, friendly populations would be killed not only by the enemy but by their friends.

It is clear that this weapon gives us an advantage that would have only limited utility for the Soviets. The same is true with respect to increased capabilities in ICBM interception: the new technology aids both sides but is more helpful to the side on the strategic defense than to the disturber power.

This strategy dictates that neutron technology should be diligently sought by the United States. If it can be kept as a technological monopoly for the free world, the advantages are obvious. If it is developed by both sides, we still retain an advantage because of strategic asymmetries. It is only if the U.S.S.R. develops neutron technology as its own monopoly that neutron weapons will upset the stability of the arms race.

Hegel's "rule of reason" never rests: the laser is coming into its own and its development is "happening," just as an avalanche, once it is formed, moves forward beyond man's ability to stop it. The laser is a God-sent for the triggering of nuclear weapons and will necessarily be used. It will improve the yield-to-weight ratio, and thereby allow higher reliability and greater accuracy; it will permit more fire power per weight of the delivery instrument, for example in the form of additional MIRVs or, conversely, make possible reduction of the size of delivery missiles and aircraft. Such a development, by the same token, would boost the range of combat aircraft and improve the capability for low-level attack. The interesting point is that the laser trigger not only greatly facilitates the construction of an all-fusion neutron weapon but would also boost the output of fission weapons. Therefore, it can be safely predicted that sooner or later all nuclear weapons will release large neutron fluxes and correspondingly will have reduced blast, heat, and electron radiation effects.

The nature of research is precisely that uncertainty is involved. Consequently, the decision to acquire such devices should be based not on the pessimism or optimism of scientists but on strategic utility. For example, in the field of controlled fusion for electric power, pessimism is very strong and thus far the pessimists have been proved right. Nevertheless, the stakes are so enormous that we are rightly pursuing the program.

\section{Yield-to-weight Ratio}
Another important aspect of the technical race between the United States and the U.S.S.R. has centered around the mass-yield or yield-to-weight ratio. Improvement in the mass-yield ratio made it feasible, first, to develop small nuclear weapons for airplanes other than heavy bombers and, second, to complement the manned airplane with airborne missiles until better mass-yield ratios were developed. The United States saw one point in missile development; the U.S.S.R., meanwhile, concentrated on very large boosters capable of lifting the then-existing H-bombs. The United States waited until smaller bombs were developed. Subsequent improvements permitted us to progress from the single-shot to the multiple-shot missile. Further improvements will facilitate the development of space delivery systems and will enhance the effectiveness of all types of delivery--ground, sea, air, ground-to-air, etc.

To take the measure of much scientific advice, it should be recalled that the advocates of the total ban on nuclear testing argued that for all practical purposes the mass-yield ratio could not be improved much beyond that attained by the United States in 1958! We now know that this assumption was entirely fallacious and that very considerable improvements have been achieved by both the United States and the Soviet Union. All the facts suggest that considerable improvements are foreseeable, both by extrapolation from known techniques and by entirely new designs that incorporate several types of nuclear reactions.

In retrospect, we know that even those scientists who were considered to be optimistic about possible improvements were far too cautious. This has happened over and over in the history of technology, as any reader of science-fiction knows, and does not necessarily mean that the superoptimists are right about the future. It does mean that we cannot assume in advance that we know the practical upper limits to processes like the yield-to-weight ratio that have very high theoretical limits, the development of which is hindered only by engineering considerations. We should never forget that the future remains unpredictable.

One immediate improvement in existing missiles will result from advances in this technology: yield per fixed weight can be increased and multiple reentry vehicles can be installed in the present carrier force, thus effectively increasing the size of the force without adding a single new carrier. Since we will soon have to make dramatic improvements in our force in order to ensure survival, the economic gains that research into mass-yield ratio improvement could bring are worth contemplating.

Improved mass-yield ratios will aid the search for survivable second-strike weapons by allowing the construction of very small missiles that retain respectable yields. This would reduce the cost of our missile force, permit smaller silos, and give us capabilities for installing superhard survivable installations. Alternatively, we can enlarge the number of missiles in the inventory without increasing the budget; and of course Soviet MIRV development must be compensated for, either through MIRV of our own, active defense, proliferation of our missile force, or new, more survivable systems. A combination of the above including manned bombers would be preferable, and by its construction each would aid advancement of the other technologies.

Improvements in yield-to-weight ratios can change the defense picture in other ways. By appropriate design of nuclear weapons, we can change the energy partition; that is, we can alter the proportionate amounts of energy given off as heat, prompt gammas, X-rays, etc. Nuclear research may produce energy partitions that make use of exotic long-range kill mechanisms to be used against enemy missiles in space. However, all these techniques are dependent upon the energy being there in the first place, and that will require better yield-to-weight ratios.

Beyond the military uses of nuclear weapons, there are applications of nuclear energy to plowshare applications, such as digging a new Atlantic-Pacific canal, blasting out harbors, mining, and constructing shelters, underground cities, etc. There is even the possibility of nuclear energy being employed to construct large bases on the moon, where "earth"-moving will be both expensive and necessary. The advantages of using low-cost nuclear techniques for constructing underground habitations are apparent.

In order to try to curtail the application of nuclear technology to weapons, extensive, long-term efforts were devoted to international negotiations, treaties, and agreements. Very much related were efforts to prevent the use of nuclear materials developed for and by commercial reactors for weapons. The International Atomic Energy Agency was established by treaty and located in Vienna, Austria. Technology for inspections and safeguards were developed for the IAEA. Non-proliferation programs were instituted by the U.S., U.K., and U.S.S.R. but with limited effects. France pursued its own path for commercial power and military weapons, developing bombs for aircraft, strategic ballistic missiles and SLBMs.

New nations joined the nuclear club. China followed much the same path as France, but also developed its own ICBMs. India exploded its own nuclear bomb to signal its arrival as a major power. In order to prevent Iraq from developing the "Islamic Bomb" financed by Libya, Israel conducted an air strike on Iraq's nuclear reactor and destroyed it.

Israel was reported to have its own nuclear bombs. Argentina, Brazil, and Pakistan have been assessed to be "on the verge" of developing nuclear weapons. In sum, non-proliferation efforts were never successful when nations decided that it was in their interests to have nuclear weapons, and they acquired the necessary technology to develop them.

The consequence for the U.S. military planner in the 1990's is that Third World countries could use nuclear weapons in wars in their region. For example, the U.S. in the 1980's pressured Pakistan not to develop its own weapon because of the fear of nuclear war between nuclear-armed India and Pakistan. (We should note that the Pakistani have another motivation, namely, to defend themselves from a Soviet invasion through Afghanistan.)

But the major focus on nuclear technology has been on strategic relations between the U.S. and U.S.S.R. The arms control theory is that if tests are banned, weapon development will stop; the arsenals will atrophy; the user will be uncertain as to the health of his nuclear weapons; and consequently they will not be used. There is a somewhat related motivation, namely, when a country believes it has a lead in nuclear weapon technology it wants a treaty to preserve that lead and prevent its adversary from closing the gap.

The history of test ban negotiations covers the entire post-war period. The harmful effects of testing in the atmosphere led to the "gentlemen's agreement" of the 1950's which the Soviets violated in 1961.\footnote{
The massive Soviet testing program of 1961 has been discussed in other chapters. A good work on the subject is Bielenson, The Test Ban Trap.}  It was followed by the Treaty of Moscow which did end atmospheric testing by the U.S., U.K., and U.S.S.R. (But not by France and China which were not signatories.) Lengthy efforts followed in the 1970's to limit underground nuclear testing which resulted in a partial ban, limiting such tests to 100 KT. Negotiations continued in the 1980's for a complete test ban which was not achieved.

A very curious situation arose as a result of the meeting of President Reagan and Communist leader Gorbachev at Reykjavik in December 1986. President Reagan proposed that the objective of stopping nuclear weapons testing be achieved another way -- to eliminate nuclear weapons entirely. Suddenly many believers in test ban theory found themselves to be "children of the nuclear age" and opposed total elimination of nuclear weapons.

The continuous drive on the part of the Soviets impacted on nuclear technology in two domains. One lay in designing and testing weapons within the partially negotiated constraints, such as limits of 100 KT's of yield. The other lay in the closely related domain of verification. Any limit on testing has the attendant requirement to determine if violations are occurring. That in turn requires verification of compliance with the treaty limits. Obviously, as the limits were decreased, the difficulty of assessing yield at lower limits increased. Thus, technology, principally the application of seismic measurements, had to be adapted to measuring yield. By the end of the 1980's that application had been very successful, giving confidence in the ability of the U.S. to verify treaty compliance.

Thus of the major lines of nuclear technology, only weapons and commercial production of electrical power survived. The strategy of nuclear weapons technology had a history of its own.

\section{Nuclear Strategy}
\begin{mdframed}[backgroundcolor=black!10]
The balance of this chapter was prepared in 1969. In the comparatively few places where it has needed revision, we have inserted parenthetical remarks.
\end{mdframed}
The Soviets have made it clear through their continued test series that they intend to perfect their nuclear weapons and improve their nuclear technology. So have the Maoists, who have made rapid nuclear progress. As of several years ago, the initiative for perfecting nuclear technology, particularly weapons, has been conceded to the U.S.S.R., and U.S. test programs have been designed largely to react to their moves. To the extent that we have had a strategy of nuclear weapons development, it has been to deplore the existence of the weapons, deny the feasibility of more useful weapons, attempt to halt testing through diplomatic means, and design test programs to be used only after the Soviet Union begins testing. There has been no attempt to seize the initiative in nuclear technology or to pursue the advantages we do have, not, of course, for the purpose of aggression and conquest but to preserve peace.

One reason for this curious lack of strategy in this most vital area has been our fear of nuclear weapons as such and our fascination with the holocaust which supposedly will end human life or, at least, civilization as we know it. Now, the authors are well aware that nuclear weapons are dangerous, and that they can be employed to exterminate a large part of the vertebrate life on this planet. However, these weapons will not simply go away if they are ignored; nuclear technology marches on inexorably, as does other technology--in fact, nuclear physics being the characteristic science of the age, nuclear technology moves far more inexorably than other sciences. More important, nuclear weapons have been of great positive benefit to the cause of peace and in the future can perform for peace again, again, and again.

\section{History of the Nuclear Race}
\subsection{Nuclear Research Requirements}
This is not a technical study, and nuclear research remains a highly classified field. Incidentally, although there is good reason for the great secrecy surrounding some of our nuclear technology, some material appears to be classified to prevent the American people from rationally discussing the problem, much as the information that U.S. planes were bombing enemy pack-trains in Laos remained officially secret. The enemy is well aware of certain information about nuclear weapons development, just as he could hardly be unaware of the fact if he were being bombed; it is the American people whose ignorance is maintained by official secrecy.

Because the subject is classified and technical, we do not attempt to detail nuclear research programs that should be funded or to specify the direction of the programs under way. We do wish to point out aspects of nuclear technology requirements that can be learned from elementary strategic analysis. Each of the areas of potential technological breakthrough shown on Chart 17 will generate enormous repercussions in both the military and the civilian technologies dependent upon them; indeed, it can be said that the exploitation of the atom has only begun, and that future nuclear research will make the military and economic environment of 30 years from now as different from the present as 1969 was from 1939.

For example, the development of earthmoving techniques combined with nuclear power plants and seawater conversion will allow construction of cities at any seacoast location without regard to natural features such as harbors. Harbors and canals can be constructed at will, water can be converted without regard to rivers, and complete underground cities with controlled climates can be constructed if there is some necessity for them. Areas with excellent climate but neither harbors nor rivers can become resorts or industrial cities. The population explosion, which is largely a result of too many people in a few sites while most of the earth remains uninhabitable, can be damped out, at least for a few generations. Even pollution of the air and the waters may be reduced through nuclear techniques, despite the fact that modern man lives in hysterical fear of pollution by radioactivity. Pollution, the unwanted child of technology, can be eliminated only through use of the most advanced technology.

In the military field, the revolution will be as great. Warfare in the twenty-first century will differ from today's war as much as warfare in the sixties differs from the German conquests in 1939. We have no choice but vigorously to pursue our nuclear research programs. There is no way to halt nuclear progress; it must not be unilateral progress by our enemies.

\section{The Impediments to Nuclear Research}
Since the crucial importance of nuclear technology is rather obvious, and the U.S. capability in this field so well recognized, those not closely concerned with the Technological War may be surprised to discover that the United States is not far ahead of the U.S.S.R. in several key fields, and may be at a loss to understand why we have not progressed more rapidly than we have. The answer lies in the nature of our scientific decision process, as well as in lack of a technological strategy, lack even of insight into the necessity for such a strategy.

The major problem with nuclear research is that many U.S. decision makers have a strong feeling of guilt about nuclear weapons and an almost neurotic reluctance to learn more about nuclear problems. The reasoning runs as follows: We have enough thermonuclear explosives to kill every man, woman, and child in the world fifty times over; why should we spend money inventing more?

The usual decision maker is not even interested in the answer to this question; he knows in advance that there is no answer.

The problem, however, is far more complicated than he thinks. Although the so-called defense intellectuals strongly suggest that this is so, mere possession of thermonuclear weapons is not enough to deter war; nor will the "just-possessed weapon" win a war if deterrence fails. Any high school biology teacher can manufacture and store in a refrigerator of medium size enough botulism toxin to kill every vertebrate creature on the globe a thousand times over, but he has not thereby stopped war, avoided defeat, or ensured victory. Deterrence weapons must be deliverable after an enemy strike--they must get off the ground, penetrate the enemy defense, and destroy the target. If technology brings forth ways to negate the defender's arsenal before it can be delivered, only one nation will be destroyed in the war.

The self-fulfilling prophecy is another serious problem in technological development. Those who are entrusted with the technical decision about a promising line of research say it cannot be accomplished; the research program therefore gets no money; and, naturally, no invention is ever made. The history of the all-fusion weapon is an excellent illustration of this tendency.

The third problem is unreasonable expectations. The new technology is expected to produce operational weapons with characteristics far beyond anything presently in the inventory, to do so making use of previously-undiscovered principles, and to accomplish this unfeasible feat cheaply and within four or five years. The nuclear airplane and early space-observation systems were treated this way.

It should be clear that a properly-designed technological strategy will obviate such artificially-created problems. If strategic analysis were to be systematically devoted to discovering technological areas in which surprise could be mounted, research funds should be invested to forestall surprise, through knowledge and anticipation, and to hedge against the possibility that the enemy will forge ahead in a crucial technique. Unfortunately, we usually ignore the problem. It is not enough for the technological strategist to bring his research and development programs into an orchestrated plan, we also need hedges against failure or success of exploratory programs and against surprises.

To illustrate what we mean by strategic analysis applied to nuclear research, we will discuss certain examples below. We do not assert that these are the only critical areas of nuclear research, nor even that they are necessarily the most important. We have tried to choose examples in which an understanding of strategic value does not depend on classified information.

\section{Conclusion}
Our examples demonstrate the importance of strategic analysis in the generation of a technological strategy. Technical skepticism can be important, and of course wasting resources on unprofitable lines of research can be disastrous. However, really vital research should not be neglected in order to achieve some kind of illusory economy. It is never economical to allow the enemy to move ahead in a decisive field in the Technological War, because the defender must then engage in crash programs that are wasteful of resources and consume far more time than would orderly development begun earlier.

When strategic analysis indicates that areas of nuclear research can lead to decisive advantages in the Technological War, and technical opinion is divided about the feasibility or time-schedule of the projected inventions, it is prudent to ensure that the enemy will not gain a decisive lead. Furthermore, if the research comes to nothing, it need not be wasted effort; not only may unexpected but highly important discoveries be made in the research effort, but through the use of misinformation and disinformation the enemy may be induced to invest equal resources in similarly unprofitable programs. A properly-drawn technological strategy will make use of this kind of deception, which has been practiced on us several times.

About the political battle over nuclear research, certain predictions can easily be made. For example, as neutron weapons are potentially of decisive importance, the Soviets will direct a propaganda campaign to hinder our technological advances in this field, holding out the prospect for arms control or even disarmament agreements which somehow are never signed or do not work out according to our expectations. This was the Soviet strategy to obtain the Treaty of Moscow, which is so cunningly worded that it affords them all the advantages. Neutron technology, like nuclear testing, will become the subject of much propaganda; still other attempts will be made to prohibit all nuclear testing, underground or not. No actual treaty will be signed, however, until the Soviets have accumulated all the data they need.

We must not fall victim to this stratagem again. The nuclear technology race is perhaps the key battle of the Technological War. We must seize the initiative, driving the Soviets to react to us rather than our reacting to them. The arguments of the technical skeptics and disarmers will be with us in the future as they have been in the past, and they will be difficult to counter. The primary question, however, is this: will we be first or second in the critical area of nuclear technology? If we are second, we may find that the gap is not closable; the results can be decisive.

\section{Notes to Chapter 7 by Dr. Francis X. Kane}
NUCLEAR TECHNOLOGY
TEXT FROM DR. KANE
1 June, 1988
Jerry, after struggling with the chapter, I finally identified its flaw. Steve assumed that a nuclear strategy meant exploiting all the potential applications, and that the Soviets constrained us by their deployment/propaganda efforts.

The fact is that we have and have had a strategy for nuclear technology. It has been very narrow; but very, very successful. The objective has been to continuously improve our weapons in spite of all constraints. The weapon technology goes hand in hand with the inertial guidance technology. As accuracy has gotten better and better, yield has gone down, and weapon effects have gone up. Furthermore, our weapons are longer-lived, have become more reliable, and have been very economical in the use of critical nuclear material.

Furthermore, as we have decreased yield without giving up military (and even improving) military effectiveness [sic] we have reduced the amount of critical material in each weapon. Thus, we are able to "mine" obsolete weapons for their nuclear material and re-use it for newer, more efficient designs.

That strategy can only be called an unqualified success.

\subsection{Nuclear Technology}
In the fifty years since Nils Bohr announced the splitting of the atom nuclear technology has grown and matured -- and become the most controversial technology in history. As we approach the end of the century, the issue is whether or not nuclear technology will continue to exist. In the immediate post-war period several landmark studies identified applications to an array of military and civil applications (See Chart 17). Most of them were explored. But at the same time there was a raging discussion of ways to limit those applications or to "put the nuclear genie" back in the bottle. That situation still prevails at the start of the last decade of this century -- new applications are being invented; new attempts are being made to prevent them.

The list of military applications explored covers nearly the entire range of propulsion and weapon systems.

Nuclear bombs grew from the 20 kiloton weapon of 1945 to the 60 megaton bomb exploded by the Soviets in 1961 and 1962, when they abrogated the "gentlemen's agreement" not to test nuclear weapons in the Earth's atmosphere. As nuclear technology matured the explosive power in bombs declined from the multi-megaton range to that of the low kiloton. Such bombs are carried by fighter and bomber aircraft, ballistic missiles (both ICBM and SLBM), and cruise missiles.

Nuclear artillery rounds were developed, deployed and modernized for battlefield operation, particularly as part of the U.S. deterrent to Soviet attack on NATO.

Nuclear air defense weapons were deployed in Europe and the U.S. Nuclear weapons for ballistic missile defense and ASATs were deployed. Nuclear depth charges were designed (Casaba Hawlyn[?]) and deployed.

In propulsion technology, nuclear powered engines were developed for long-range bombers (the Camel[? Comet?])' nuclear powered cruise missile; nuclear reactors for ships, both surface and submarine; nuclear propulsion for spacecraft (SNAP) was used; and nuclear power for space stations and lunar bases were designed. A design to propel large satellites by a chain of nuclear explosions (Orion) was developed but never implemented. The Soviets for their part have developed and orbited many nuclear reactors.

On the commercial side, nuclear power for the generation of electricity became the mainstay of France and other countries. Nuclear explosions for peaceful purposes such as building canals were actively considered for a decade and dropped.

All these applications were constrained by fear; fear of accident, pollution and unknown effects, and the overriding dread that the use of even one nuclear weapon would lead to the end of mankind. To allay fears, emphasis was placed on safeguards and constraints. Nuclear weapons on aircraft, for example, were controlled by Permissive Action Links (PAL) so that they could be used only on authority by[?] responsibility civilians. Most of the applications for propulsion were dropped because of the impossibility of safe military operations. The nuclear-powered aircraft was to have been flown in remote areas of Utah and a special hangar was built for it even though the program never survived the design stage. Of those propulsion applications only the nuclear reactor to power submarines survived and matured.

In like manner, nuclear plowshares never became a real program. Nuclear power for generation of electricity survived, albeit controversy surrounded individual plants and caused delays in construction, cancellation of programs[?], and even abandonment of plants.

Very much related was the issue of disposing of nuclear waste materials. The search for suitable sites dragged on for years, hindered by concerns for pollution of the water supply and other health hazards.

Nevertheless, invention continues. New ways to focus energy produced by nuclear explosions were developed. One application was postulated for the X-Ray as a source of power to destroy enemy ballistic missiles. Tailored weapon effects for discriminant[?] employment were developed for the "Safeguard" bird[?] program[?] and battlefield weapons, and the most controversial of the[?] inventions was the "neutron bomb" which could kill enemy forces by enhanced radiation and not produce damage to material.

In order to try to curtail the application of nuclear technology to weapons, extensive, long-term efforts were devoted to international negotiations, treaties, and agreements. Very much related were efforts to prevent the use of nuclear materials developed for and by commercial reactors for weapons. The International Atomic Energy Agency was established by treaty and located in Vienna, Austria. Technology for inspections and safeguards were developed for the IAEA. Non-proliferation programs were instituted by the U.S., U.K., and U.S.S.R. but with limited effects. France pursued its own path for commercial power and military weapons, developing bombs for aircraft, strategic ballistic missiles and SLBMs. China followed much the same path but developed also its own ICBMs. India exploded its own nuclear bomb to signal its arrival as a major power. In order to prevent Iraq from developing the "Islamic Bomb" financed by Libya, Israel conducted an air strike on Iraq's nuclear reactor and destroyed it. However, Israel was reported to have its own nuclear bombs. And Argentina, Brazil, and Pakistan were assessed to be "on the verge" of developing nuclear weapons. In sum, non-proliferation efforts were never successful when nations decided that it was in their interests to have nuclear weapons, and they acquired the necessary technology to develop them.

The consequence for the U.S. military planner in the 1990's[?] was that Third World countries could use nuclear weapons in wars in their region. For example, the U.S. in the 1980's pressured Pakistan not [to?] develop its own weapon because of the fear of nuclear war between nuclear armed India and Pakistan. (We should note that the Pakistani have another motivation, namely, to defend themselves from a Soviet invasion through Afghanistan.)

But the major focus on nuclear technology has been on strategic relations between the U.S. and U.S.S.R. The arms control theory is that if tests are banned, weapon development will stop; the arsenals were[?] atrophy; the user will be uncertain as to the health of his nuclear weapons; and consequently they will not be used. There is a somewhat related motivation, namely, when a country believes it has a lead in nuclear weapon technology it wants a treaty to preserve that lead and prevent its adversary from closing the gap.

The history of test ban negotiations covers the entire post-war period. The harmful effects of testing in the atmosphere led to the "gentlemen's agreement["?] of the 1950's which the Soviets violated in 1961. It was followed by the Nassau Treaty which did end atmospheric testing by the U.S., U.K., and U.S.S.R. (But not by France and China which[?] were not signatories.) Lengthy efforts followed in the 1970's to limit underground nuclear testing which resulted in a partial ban, limiting such tests to 100 KT[?]. Negotiations continued in the 1980's for a complete test ban which was not achieved.

A very curious situation arose as a result of the meeting of President Reagan and Communist leader Gorbachev at Reykjavik in December 1986. President Reagan proposed that[?] objective of stopping nuclear weapons testing be achieved another way -- to eliminate nuclear weapons entirely. Then many believers in test ban theory found themselves to be "children of the nuclear age" and opposed total elimination of nuclear weapons.

The continuous drive on the part of the Soviets impacted on[?] nuclear technology in two domains[?]. One lay in designing and testing weapons within the partially negotiated constraints, such as limits of 100 KT's of yield. The other lay in the closely related domain of verification. Any limit on testing has the attendant requirement to determine if violations are occurring. That in turn requires verification of compliance with the treaty limits. Obviously, as the limits were decreased, the difficulty of assessing yield at lower limits increased. Thus, technology, principally the application of seismistic[?] measurement, had to [be?] adapted to measuring yield. By the end of the 1980's, that application had been very successful, giving confidence in the ability of the U.S. to verify treaty compliance.

Thus of the two major lines of nuclear technology: Weapons and other applications, only weapons and commercial production production of electrical power survived. The strategy of nuclear weapons technology had a history of its own.

In the initial period the focus was on nuclear weapons design and production. Two major designs were pursued: Implosion and insertion. The objectives in weapon design were efficiency and safety. As for efficiency, there were two drivers[?] -- improving the yield to weight ratio and decreasing the amount of critical material used. Safety aspects concentrated on the bombs themselves, including extension of life of the weapons, and maintenance of reliability. During this period also, an entirely new weapon design -- the hydrogen or "H" bomb.

Principally under the influence of the ICBM program, design shifted from weapon development and production to that of weapon systems. The marriage of a small weapon with a rocket booster lead[?] the way to an integrated approach. The first major product was the MIRV'ed ICBM, but others were found in the SLBM, field artillery, and tactical fighter delivered weapons.

Beginning at about the same time, other technologies, principally electronics, began to play a major role in nuclear weapons. The internal guidance system of the Post-Boost Vehicle which carries the MIRV's greatly improved the accuracy of weapon delivery. That technological innovation meant that the yield of the individual weapon could be reduced while still maintaining weapon effectiveness as measured by SSPK (Single-Shot Probability of Kill).

Another important application of electronics came from the political requirement for absolute control of each weapon. The concern was that aircraft-carried weapons could be employed by the aircraft crew on their command. Thus, inadvertent, accidental, or deliberate but unauthorized release could occur and nuclear war could result. Consequently, Permissive Action Link[s?] (PAL's) were designed and installed on nuclear weapons. For ICBM's in their silo's, a "turn-key" system was installed so that no one crew member could launch a missile, because two members would have to "turn their keys" in a prescribed sequence and on receipt of a coded message in order for ICBM launch to occur.

Implementation of the INF will remove the newest, most effective nuclear weapons from the U.S. stockpile. This is a reversal of prior treaties which resulted [in?] or permitted removal of older, less efficient weapons while retaining the most modern ones. One of the effects of the INF is thus to increase the average age of the U.S. nuclear weapons stockpile.

In the decade of the 1990's, nuclear weapon technology will see a new phase -- transformation to "wizard" weapons. During the war in Viet Nam advanced guidance technology, notably lasers, was adapted to World War II conventional bombs to make them more effective. Heroic feats of air delivery against selected elements of the power plants in Hanoi with a CEP of 14 feet. Similarly, such guidance systems will be adapted to nuclear weapons to produce in effect zero CEP weapons.

Optical guidance, map matching, radar guidance including laser radar are available. The weight of such guidance systems will be measured in ounces, not pounds; their mass will be practically zero also. These application to nuclear weapon design with still greater improvements in yield to weight ratios, will result in new weapon capabilities. Such advances will cascade into small, more effective weapon systems.

Given the advances made in the 1960's in discriminate nuclear weapons, the ultimate will be highly effective performance with controlled energy release from small weapon systems.

On the other hand, nuclear weapon technology can be applied to ballistic missile defense by X-Ray lasers which can have destructive effects at very long distances in space. Whether based in space or on the ground with relay mirrors they can achieve such effects instantaneously. Such lasers and other "speed of light" weapons will be possible in the next century.

But as has been the history of nuclear technology, the development of "wizard" nuclear weapons and "speed of light" weapons will be constrained and perhaps prevented by policy decisions. And those decisions will flow from the continuing fear of the atom, fear which has retarded many potential applications.

A key factor in the evaluation of policy will continue to be the Soviet drive for a total ban on testing. A comprehensive test ban would mean the end of nuclear technology.
