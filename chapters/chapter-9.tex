\chapter{The Prevention of War}

\section{Why Wars Are Not Fought}
The primary stated objective of the United States is to preserve the state we call peace. The Strategic Air Command, which controls more power than all other military organizations throughout history, has as its motto, "Peace Is Our Profession." Our diplomatic machinery is geared to negotiations for peace, and our alliances are defensive. If intentions alone would produce peace, we would have it.

Our pursuit of peace is complicated by two important factors. The first of these is confusion about the meaning of peace. In legal terms we are at peace whenever the Congress has not declared war. Yet we can be actively engaged in a shooting war, and as this book has shown, the Technological War goes on, without regard to legal niceties, as a permanent conflict.

Many of our international legal institutions were conceived and solidified at a time when there was a far greater distinction, even a chasm, between peace and war. In those nearly-forgotten times, when nations went to war they acquired "rights of belligerency," which they could invoke against other nations. Perhaps today there should be some recognition of the rights of cold-war belligerency and of the Technological War. If laws ignore the real situations in which people live and reflect fictitious assumptions, the legal order is decaying and society becomes vulnerable. The point is not to curtail rights, freedom, and democracy, but to keep them working during critical times and to provide a reasonable legal basis for the requirements of security.

The other impediment to the achievement of peace is the paradox known since Roman times: "If you would have peace, prepare thou then for war." The unprepared rich nation without armed allies has never survived for long. Wealthy nations have ever been forced to depend on their readiness for war to preserve peace and survival. Yet history seems to indicate that the greater the state of conventional armaments acquired, the greater the chance for war; and consequently many well-meaning people in the contemporary United States believe that the surest road to peace in the nuclear era is arms limitations which may hopefully lead to disarmament.

This misconception stems from an insufficient appreciation of the modern era. The nuclear weapon has changed the nature of warfare by providing the defensive power with a capability to deny victory to the aggressor, even if the aggressor has successfully destroyed all but a small fraction of the defender's military forces. Unlike conventional weapons, nuclear weapons do not increase the chances of war as both sides acquire them.

This is so because mutual increases in the nuclear power available to the superpowers do not cause mutual increases in their expectations of victory. In fact, the opposite is true. All but madmen recognize that as mutual capability for destruction increases, the possibility of gain through initiating that destruction becomes smaller. Whatever the effect of arms races in conventional weaponry, two-sided arms races in the nuclear era have a stabilizing effect in so far as the outbreak of total war is concerned.

Wars are fought because decision makers conclude that they will be better off after the war than they would be if they did not engage in them. This has been true whenever rational decision processes have governed the war decision. The calculation of success is not a matter of objective reality only, but is in large measure a process in the mind of a strategist controlling military power. It is not sufficient merely to be sure that no one can win against you; all those who might attack must be convinced of it as well. In addition, the definition of win may be different for a potential aggressor than for a popularly-elected chief of state; and it is necessarily different for one aggressor fighting for nationalism or nationalist imperialism than for another aggressor who fights for international communism or Communist imperialism.

The calculation of chances of success is spoiled by uncertainty; indeed, uncertainty about the outcome of a war is a powerful deterrent in the absence of clear indication of the enemy's power. When both sides are engaged in nuclear arms research and deployment, neither will be very certain that he has won the Technological War and can engage in nuclear strikes or blackmail. It is when one side drops out of the race, giving the other a clear shot at technological supremacy, that a strategist can begin to plan on terminating the contest by a nuclear strike.

Deciding on war is also a matter of will, which is essentially willingness to take risks and troubles. Virtually always, the will factors are vastly different for the two parties engaged in conflict. Circumstances that would cause one to initiate a war might not tempt the other. Dictatorial regimes are notoriously generous with human lives; democratic governments fear casualties and usually fight only, and frequently belatedly, to preserve their own security. The will factors change when political systems are on the rise or decline. A dictatorship, for example, is optimistic early in its youth--it may combine determination with caution, or it may be exuberant. But it reacts differently when it is senescent: it then has the rationality of despair, and it may prefer a last chance through war, and even defeat on the battlefield, to ignominious overthrow. World War I would hardly have occurred if Russia, Austria-Hungary, the Ottoman Empire, and China had not been decaying. There will be decaying regimes in the future. In particular, the Communist dictatorships won't last, but the period of their departure will be difficult and dangerous, and the rationality of their last leaders may be influenced less by probabilities of success in a nuclear contest, than by considerations of last chance stratagems, envy, revenge, and pure hatred. There is no such thing as equal rationality for all.

Calculation of military results, then, is only a part of the decision to go to war. Strategy serves as a tool for the political decision maker, and the calculation of military success sets the probable price in blood and treasure that must be paid in war. The political objectives are the factors that determine what price a government is willing to pay; and these objectives are not set in absolute terms. If world domination is the objective, then no price is too high provided that the rulers of the aggressor nation will survive and remain in control and all other countries will be reduced to impotence. Conversely, if the probable result of the war will be the overthrow of the ruling structure, no victory, no matter how cheap in lives and property, is worth the winning.

The calculation of political objectives, the disparity of objectives between the major powers of the present world, and the state of the Technological War are the primary factors in the decision to begin wars. However, they have received less attention than mathematical calculation of military factors, which has become prevalent. It is supposed by many that even if the political objectives of the U.S.S.R. can never be understood with certainty, at least the military calculations on which they must base their decisions can be replicated with some assurance. This assumption needs to be examined in some detail.

\section{The Nature of Strategic Decisions}
Military calculations must take into account numerous objective factors such as force levels, weapons performance, defense systems, and the like. A strategist's advice will thus be based in part on his predictions of the material factors of battle. Success in war, however, is dependent on the competence of generals and commanders as well as on their equipment. Bad generals can lose wars even though they have the best armies, as witness the performance of the "finest army in Europe" (that of France in 1940), while good generalship can more than make up for numerical and even technical inferiority. The strategist calculates his chances of success not from statistics but from an operational plan.

His plan must take into account the quantitative factors, but it will also seek to create and exploit opportunities. War is a matter of will as well as equipment, and paralysis of the enemy's will through surprise is one of the most successful of all techniques. In war, there are real uncertainties as well as statistical probabilities. Many factors can never be quantified. The strategist is dealing with the enemy's creative force, and will counter it with his own. The calculation of destruction by means of slide rule and computer can only be a part of this process.

If this appears vague and uncertain to those more used to scientific calculation, it is because war is uncertain. War is after all an operation primarily against the will of the opponent. In some few cases, of course, the opponent is so reduced in capability that his will is no longer an important factor, but most wars have ended long before the loser's capability to damage the victor was destroyed. The great losses have occurred after surrender, in pursuit or by deliberate execution of prisoners, rather than before the decisive moment of battle. But once a combatant has lost the will to fight, his means are unimportant; and often this failure of will has been caused by surprise, by the opponent doing the unthinkable, and by so doing producing overwhelming paralysis.

It is generalship, not a calculus of forces, that decides the outcome of wars. A good general identifies opportunities to paralyze the will of his opponent and exploits them. Indeed a good strategist creates such opportunities. Generalship operates against the enemy's forces as well, of course, but even then the war is primarily against the will. When the enemy ceases to fight, the war is over, no matter that the vanquished may actually be stronger than the victor--as Darius was at Arbela. Success in war is above all dependent on generalship; it is not that objective factors such as force relationships do not count, but that generalship is far more significant. And generalship means optimal utilization of available strengths and out-thinking the enemy. Bad generalship is a repeat performance, whereas good generalship is an act of creation, hence unpredictable by either side.

Historical experience is explicit on the crucial impact of generalship: a bad general can lose despite superiority in material force and a good general can win despite considerable inferiority. Given reasonable means, and sufficient strike and reserve forces, so that the aggressive side would not be crushed even if mistakes were committed, the aggressor will calculate his chances of success not on the basis of statistics, but an operational plan, as we have pointed out. If he is a sound planner, his plan will take into account all the qualitative factors, but go beyond them to employ surprise in all elements such as strategy, technology, tactics, training, direction, concentration, and phasing. If the would-be aggressor estimates that the defender will be unable to anticipate his plan and will not have ready countersurprise operations to upset the implementation of the operation plan, he will conclude that his chances of success are high.

It is very important to understand that in these matters the calculus of generalship is far more important than the calculus of force relations. A homely example would be an investor who plays the stock market through mutual funds and thus essentially benefits by or loses from the overall movements of the market. Such an investor can calculate his probable successes on the basis of curves depicting the performance of the market in the past. However, the most successful investors operate both with and against the market. In like manner, a good strategist can identify special situations or opportunities and work out a scheme to take advantage of the openings. Naturally, in a war where there are many opportunities there are only a few that hold great promise of massive success, even if they are exploited with the greatest skill. Furthermore, good opportunities may be fleeting and there may not be enough time to exploit them properly. On the other hand, the strategist who possesses large resources, like the market operator controlling large funds, can create suitable opportunities.

These observations apply to both the offensive and the defensive strategist. Success always depends on more than the resources in hand. It results from a clear knowledge of the objectives to be gained by the particular strategy and from seizing the initiative in carrying out the strategy.

Whether planning aggression or defense against aggression, the strategist must calculate the results of the clashes of forces. He must always remember that he is dealing with human action, the essence of which is creativity. As a consequence, he knows he is grappling with uncertainties, with the basic uncertainties that result from the creativeness of the adversary.

\begin{mdframed}[backgroundcolor=black!10]
In these days of high speed computers and complex computer simulations, we often forget that strategy comes from a strategist, and victory and defeat are events in the minds of the victor and the defeated. We shouldn't. JEP 1985
\end{mdframed}

\section{Offense and Defense}
In this interplay of creativities, the aggressor has certain advantages that come from his position. The decision to attack is his. Thus, he knows when hostilities will begin. The defender cannot have this certainty. Every moment can be the moment of the attack. To heighten the effects of his blow, the attacker strives for surprise in as many elements of his strategy as possible. One of the problems of the defender is to prevent his being the surprised. This increases his needs for information about the intent of the enemy and requires him to expend resources on being constantly ready.

The attacker can build his plan for aggression around the availability of a decisive weapon. This can take the form of a technical surprise for the defender, but it need not. If the aggressor calculates that the defender cannot counter his new advance in time, he can make his decision on the basis of this crucial superiority.

In the present age of total conflict, the aggressor can manipulate the many facets of his strategy to produce a wide variety of threats and opportunities. Political warfare, economic warfare, propaganda, the struggle for technological supremacy, diplomatic maneuverings, subversion, and military operations, taken together or individually, give the aggressor many opportunities. The defender, for his part, must provide a total defense against all these forms of conflict. Most important, he must avoid being second in technical advances that can lead to a decisive military advantage.

The defensive strategist is not without advantages on his side, provided that he does not passively wait for the blow. he can take initiatives to gain and maintain a position of superiority in the various forms of conflict. By having such superiority, he prevents the aggressor from finding the moment for the attack. The defender can plan and execute his own surprises against the would-be aggressor. The combination of initiative and surprise on the part of the defensive strategist produces the creative uncertainty that negates the advantages of the attacker. It is a military truism that strategic offensive and tactical defensive is often a superior position. Sun Tzu says, "Take what the enemy holds dear and await attack."

It is axiomatic that the objectives of the attacker and defender are asymmetric. Thus, the initiatives they take and the advances they make need not and probably should not be of the same kind. They should be chosen for their potential ability to reach the objectives of the strategy.

\section{The Modern Strategic War}
The strategy of the United States, and indeed free world strategy, is defensive. We seek no political, economic, or territorial aggrandizement. We do seek to prevent war. These objectives are clearly in direct opposition to those of the Communist bloc. They seek world domination. They create opportunities to use warfare to attain it.

\begin{mdframed}[backgroundcolor=black!10]
This was written long before the falling dominoes in Indo China or the Soviet invasion of Afghanistan, and indeed before any but a handful of Western analysts understood the importance of the ideology of conquest in justifying the dominance of the tiny ruling group (known as the nomenklatura) over the Soviet masses. JEP, 1985
\end{mdframed}

We should recognize clearly that our defensive strategy must include initiatives and surprises. Ours need not be a reactive strategy. In fact, the struggle for technological supremacy makes a reactive strategy a most dangerous one. Waiting for clear indications of Soviet initiatives can prevent us from acting in time. We must be constantly on the initiative to anticipate their moves and to create situations to which they must react.

\begin{mdframed}[backgroundcolor=black!10]
General Daniel O. Graham first proposed Project High Frontier as a "strategic sidestep into space". In 1981 a Strategic Defense Initiative -- STAR WARS® in popular parlance -- was urged by the Citizen's Advisory Council on National Space Policy as a means of seizing the initiative in the Technological War. JEP, 1985
\end{mdframed}

In the past, we used technology to overcome the advantages of the Soviets both in the resources they control and in the initiatives we conceded them. We succeeded in negating their quantitative superiority by the qualitative advantages we acquired through a superior technology.

The circumstances today are radically different. The Soviets have challenged us in technology. They have enlarged the spectrum of conflict, taking advantage of the inevitable new struggle, the Technological War. No longer are we free to follow an independent course in implementing our strategy. We must meet the technological threat as well as the threats in other forms of conflict.

\begin{mdframed}[backgroundcolor=black!10]
Since this was written the Soviets have developed, among other things: neutral particle beams; laser weapons; mobile medium to long range ballistic missiles; satellite destroyers; etc., etc. JEP, 1985
\end{mdframed}

This form of warfare has become crucial. A technical advance can lead to a decisive military advantage. It is not enough for us to continue past approaches to our total strategy. The strategist must recast his thinking if he is to make his defensive strategy effective. He must find avenues for the initiative in technology. He must prevent the would-be aggressor from attaining a clear advantage in any aspect of technology that could be translated into a decisive military advantage.

Broader horizons are needed in another aspect of the problem of the defensive strategist. Planning methodology and decision processes reflect the past situation. They are no longer adequate. The time has come to break the shackles of science on planning methodology. We need to rehumanize planning and strategy. This process will have a direct impact on decision making. Decision makers can no longer find refuge in the alleged certainties and probabilities that past planning provided them. We are now in an era of creative, dynamic uncertainty. We must have a strong defensive position. But we must also create strategic diversions, feints, deceptions, and surprises.

Only in this way can the defensive strategist ensure that the attacker will choose not to strike. A viable strategy poses insurmountable problems for the aggressor.

\begin{mdframed}[backgroundcolor=black!10]
In 1983 the United States began, haltingly, to shift toward a strategy of Assured Survival rather than Assured Destruction. Some of the principles set forth in this book were applied, in may cases by officers who had been assigned the first edition in their service academies and war colleges.
By 1989 the benefits of this strategic shift became obvious, as the Soviet Union found itself short of resources, and doomed to watch its expensive Strategic Offensive Forces becoming, in President Reagan's words, "impotent and obsolete." Their first response was to redouble their efforts to achieve technological victory. The result was a great strain on their economic resources; ultimately a strain they could not bear. Shortly afterwards the Soviet Empire in Eastern Europe began to unravel. Then the Berlin Wall came down.

This is not to say that the Cold War is over; it is certainly not to say that the Technological War is finished. That can never be. The silent and decisive war will continue well past the end of this century. It does mean that we have won a major battle in the Technological War. We have not yet exploited that victory with technological pursuit.
\end{mdframed}

\section{The Effect of Nuclear Weapons}
Nuclear weapons have not changed the nature of strategy; however, they have introduced new complications, just as they have introduced new opportunities. The major new opportunity is that the strategist, once he decides to strike, can apply far more power, over larger areas, than could any of his predecessors who fought with low-energy weapon systems. The main new complication is that the defender, even though deprived of a large portion of his initial strength in the course of the first battle, would still retain enormous firepower to hurt the attacker far more dangerously than it was ever before possible for a defensive force to do. This residual force can be used against the attacker's population, industry, urban areas, government control centers, and armed forces, provided that the defender has not acquired active and passive defenses, and the defender's will to retaliate has not failed. The scope of war has grown to include entire undefended populations, not merely military forces.

Also, it is easily conceivable that under some circumstances the attacker, although he may defeat the defender, may achieve only Pyrrhic victory. Or, even if he achieves an unqualified victory, he would not enjoy the fruits because he has paid an excessive price. In fact, he may have lost his country to the blows that his defeated adversary was still able to inflict. Nuclear weapons have not worked totally to the advantage of the attacker.

It would be a grave mistake to assume that this particular strategic problem is new. Even if it were new, the significant aspect is whether the danger of devastating retaliation would prevent war. To put it in different terms, the question is whether such a hazard would prevent the aggressive strategist from planning for war in a rational manner. Obviously a great deal of the strategist's mental effort must be devoted to the security of the aggressor's homeland. If the defender can be induced to leave his weapons unused, the aggressor can still achieve decisive victory.

In order to prevent aggression, the defender must seek safety in strength. He must seek superior technology, modern weapons that can survive attack, and engage actively in the Technological War. He cannot rely on agreements, planned weaknesses, or minimum strength.

In the last analysis, superior strength remains the most reliable insurance for survival of the defender. The strategist of the superior power has some chance of predicting what his enemy might do; the strategist of a greatly inferior power can only hope. A defensive strategy aiming at superiority in power offers the only dependable hedge against errors in planning.

\section{Force Levels in the Nuclear Era}
It is clear that as the armaments race takes place on high force levels the aggressor will be hard-put to achieve decisive superiority. The conclusion to draw from this is that relatively low levels of nuclear power are a chief prerequisite for nuclear attack. This is especially true in a period when cities have not been fully dispersed and populations have not been cannot be effectively protected by a program of active and civil defense. Low levels of forces in being are more of a danger to the United States than high levels--fears of genocide and the arms race notwithstanding. This is an important finding, which casts a very disturbing light on the recent history of U.S. armaments and armament negotiations.

Of the two belligerents, the one who is able to continue the war beyond the initial strike will have an enormous advantage because the side that does not have this capability will cave in morally and will be unable to reconstitute its force. Such a capability can only be provided by vigorous pursuit of technology, including the design and the deployment of weapons.

The survival force is one key to security. As long as the U.S. has secure weapon systems that can ride out the initial and follow-on strikes, the U.S. will be able to deter any rationally planned attack. The in-being power that is still effective after the battles are over will determine the final outcome.

\begin{mdframed}[backgroundcolor=black!10]
Project 75 concluded that due to increasing Soviet missile accuracy, by about 1980 the Minuteman force would no longer be able to ride out a full Soviet first strike, and that sometime thereafter the US would be forced to choose between active defense and launch on warning. Launching Armageddon on early warning of attack is not an attractive alternative. As warning times grow shorter, the US is forced seriously to consider computerizing the launch decision. This is even less attractive. Fortunately, the advent of SDI changes the equation. See below.
\end{mdframed}

Our defensive strategy requires us to have a survivable force. Hardening, dispersal, mobility, and concealment contribute to survival, but they are supporting strategic themes. The single most important element of our defensive strategy is to have in being a clear superiority in effective and reliable numbers. This is the one factor in the strategic equation that is most easily understood and the one the enemy is least likely to misunderstand. Numerical superiority on our side is necessary to convince the aggressor not to strike.

A would-be aggressor, if he were to act rationally, would realize that he cannot cope with high force levels. Therefore, he must make an attempt to bring forces down to a level where he can fight nuclear war--especially if through clandestine armaments of his own he achieves an enormous superiority. It is clear that the aggressor is not particularly perturbed by high force levels of his own, let alone by relative superiority, but is disturbed by a high force level owned by the defender. His problem, therefore, is to achieve a substantial quantitative superiority. To achieve this goal he must persuade the defender to be content with moderate strength.

Another reason why the aggressor needs the low level of forces is that decisive increments in strength are difficult to conceal if they have to be produced within the framework of high force levels. This means that psycho-political strategy is an integral part of nuclear strategy, first to achieve some sort of reduction of armament levels, then to provide a cover to conceal the aggressor's armaments and third, to facilitate nuclear blackmail and prevent retaliation.

\begin{mdframed}[backgroundcolor=black!10]
This was entirely true in 1969 when it was written. Since then the Technological War has continued. ICBM accuracies have been greatly increased to the point where it is nearly impossible to deploy survivable fixed-site missiles without active defense.
However, we now have the economic and technological resources to defend our SOF. The same defenses will also greatly reduce damage to our cities if deterrence fails. Practical strategic defenses allow reduction of strategic offensive forces without consequent loss of stability. In the absence of those defenses, though, the above arguments apply with full force.
\end{mdframed}

For the foreseeable future, strategic stability depends on both numbers and defenses.

\section{Security Through Arms Control}
The unending process of armaments has often been criticized as the greatest waste of which mankind is guilty. It is true that if both sides stay in the race and run well the world situation will remain stable and no war will occur; the weapons will then be said to have been wasted. The arms control argument is that if both sides agree not to engage in arms races, peace would be preserved effectively and at far less cost.

However, we have already seen that by lowering the levels of destruction war would bring, reductions in arms make war thinkable and more therefore likely. This, it would seem, is one major argument against arms control and disarmament. However, it is hardly the only such argument.

Since technology is dynamic, no one can agree to stand still. Force relationships change in the course of armament cycles despite the best planning possible. Sudden accretions of military power can come to a side not even expecting them. New technologies create new power.

It is true that the tempo of this eternal race can be accelerated or slowed. Aside from the technological factors that often determine this tempo, the speed of the process is largely set by political factors, including strategic intentions. If no disturber power is at work, the tempo will slacken almost automatically. If there is a disturber power, explicit or tacit slow-down agreements are at best highly unreliable and temporary. The side that takes the risk of slowing down unilaterally will soon be punished.

The history of disarmament agreements teaches an explicit lesson: international promissory treaties are almost invariably broken and are therefore an utterly undependable instrument of national security.

The fundamental reason for the defender to stay in this expensive race, and to run hard in it, is to stay alive and not allow the would-be attacker to achieve such an advantage that he might be inclined to break the peace and impose his will on the naive and gullible defender.

So long as the defender must stay in the arms competition, he does not really have the option of running a selective race. He cannot leave open any geographical or technological flanks, or the opponent will take advantage of his opportunities. Thus, the United States does not really have the free choice of saying that it will stop communism in Europe and defend the East Coast, but ignore Communist advances in Asia. Nor can it say that it will maintain offensive nuclear weapons but not acquire defensive ones, or that it will try to be strong in inner space but will assume that outer space is of no military relevance. Least of all can the United States entrust its security to so-called disarmament treaties, not because it must necessarily and always presuppose bad faith on the part of other nations (although sometimes it must make precisely such an assumption of bad faith), but for the far more elementary reasons that (a) reliable inspection of disarmament agreements is unfeasible, (b) that enforcement against treaty violations requires war, and (c) that disarmament agreements apply to weapons already in existence, but will be speedily outdated and be rendered irrelevant by new weapons, the characteristics of which were unknown at the time the treaty was written.

\section{Security in the Modern Era}
As we have seen, security cannot be guaranteed by Soviet intentions; not only do Soviet theorists predict inevitable victory by the U.S.S.R., and Soviet generals hasten to install the latest weapons, but, even were we convinced that the U.S.S.R. is ruled by men who have lost their aggressive drives, there is no guarantee that a new Stalin will never again take power.

Security cannot be guaranteed by passive measures. The most modern force purchased at enormous cost will become obsolete in only a few years. Security cannot be guaranteed by agreements to halt the Technological War; the stream of technology moves on without regard for our intentions. The only way to guarantee security is to engage in the Technological War with the intention of winning it. It is as true today as in Roman times that "If you would have peace, prepare thou then for war."

Regardless of the enormous effects of modern weapons, organized brainpower remains the strongest and ultimately decisive factor. The experience of Vietnam, the test ban, and the Sputnik have shown that we do not excel in that department. It is not that we lack intelligent people but that we lack an effective organization through which we can optimize our brainpower and collective memory. On the contrary, the more we have overorganized, the more we reduced brainpower and the more we forgot. Secretary McNamara even organized strategic amnesia.

We must decide to engage in the Technological War, and we must create the planning staff to guide us in this decisive conflict. To do anything short of this is to risk national suicide. At the same time, we must preserve the values that make our society worth defending; we cannot contemplate ending the Technological War by destroying our enemies without warning. Our goal is the indefinite preservation of peace and order, and our hope is that in such an environment the root causes of conflict will slowly wither.

The era of Technological War has not ended conflict, and that millennium may never come. Technological War does, however, have the advantage of being relatively peaceful, so long as the stabilizer powers remain strong. Despite the greatest threat Western civilization has ever known, since 1945 the amount of blood shed to preserve the peace has been quite small--smaller than that shed on the highways. The Technological War can be kept silent and apparently peaceful so long as we continue to engage in it successfully.

Despite fashionable rhetoric, history shows that American supremacy brings relative peace and stability to the world; where the U.S.S.R. has enjoyed local superiority the results have been quite different. American success in the Technological War is the primary prerequisite for the preservation of world peace.

In 1985 Secretary of Defense Caspar Weinberger said, "To the extent that we in the United States desire true peace with freedom, peace based on individual and sovereign rights and on the principle of resolution of disputes through negotiation, we must acknowledge and follow our interests in creating conditions in which democratic forces can gain and thrive in this world. A world not of our making, but a world in which we must fight to maintain our peace and our strength. And a world in which the very best way to maintain peace is to be militarily strong and thus deter war." Of course we agree.

The dramatic benefits of that strategy became apparent in 1989. It is vital that we understand that despite the events in Eastern Europe the Technological War is not over. It has only changed fronts.

