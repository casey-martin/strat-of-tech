\chapter*{Preface}
\addcontentsline{toc}{chapter}{Preface}

A Disquisition on the Strategy of Technology, originally in View 30.

It is probably worth keeping a bit closer to the book.

I mentioned Strategy of Technology in a recent letter to subscribers, and discovered that some had never heard of it. That goes to show that this web site is more complicated than it ought to be, and suggests that I need to do more work on organization to let people know what's here. Herewith a short explanation:

Fair warning: this is done informally and from memory, and I may have one or two details wrong.

Strategy of Technology was written in the 60's and published in 1970. The authors of record were Stefan T. Possony and Jerry Pournelle, and the book, long out of print, was published by Dunellen The University Press of Cambridge, Mass., which no longer exists. There was in fact a third author, Francis X. Kane, Ph.D., (Col. USAF, Ret'd) then the Director of Plans for USAF Systems Command.

The book was a success d' estime: that is, it was quite influential, but sold something under 20,000 copies, and went out of print when the publisher vanished. For a while it was a textbook in all three Service Academies and remained so for several years at the Air Force Academy in Colorado Springs. It was also used at the Air War College in Alabama and the National War College, and it's my understanding that Xerox copies are used (with our permission) in some classes at the war colleges to this day. Strategy of Technology was very much a book for the Seventy Years War (or Cold War if you like); although the principles remain true and important, all the examples are pretty well drawn from that conflict and specifics are directed to weaknesses in the nomenklatura system that governed the USSR in those times.

Over the years we rewrote some of the chapters and published them in various places including my own THERE WILL BE WAR series (books that were about 3/4 science fiction but which contained significant non-fiction essays on military history and principles). Dr. Possony had a disabling stroke in the mid-1980's and died shortly after the Cold War ended; he was lucid enough to know that the USSR was brought down, and that he had been a key player in that game. As one of the authors of the seminal THE PROTRACTED CONFLICT (with Robert Strausz-Hupe and William Kintner) as well as STRATEGY OF TECHNOLOGY, and in countless other ways, he had a major influence in winning the Seventy Years War. In my judgment we would not have won the Cold War without him; he was one of the great men of this century although few have heard of him today.

The book has been out of print for years, and when this web site began I was urged to make copies available here, which I did. I have posted the "revised" edition, which contains most of the first edition, prefaces, and some extensively reworked chapters done mostly by Kane and myself, although Stefan had a hand in some of the earlier revision, and we did discuss the later ones with him. After his stroke he remained aware but had great difficulty in communication, which produced extreme frustration as he tried to convey important thoughts that came out incoherently; a very painful situation for all concerned.

The html code which presents the book with extensive notes was done by professionals working as volunteers, and has some minor flaws, (I would be grateful to anyone who can correct them) but the book can be read here. I have been urged to make it available in Acrobat pdf format, but I have never had the time to do so. Perhaps one day.

When we put it up here we called it an experiment in shareware, and I asked that if you read the book you send me a dollar; a dollar bill in an envelope will do. Some have also added a couple of dollars to checks sent as subscriptions to this web site. Over the years that has amounted to a couple of hundred dollars, and at his birthday party a week or so ago Dr. Kane and I agreed that rather than divide this small sum, I'll just send it all to Dr. Possony's widow, who still lives in Mountain View.

Regina Possony was a survivor of Stalin's prison camps (they met in the United States after both had fled). She was born in Berlin and her father was an influential Communist politician who fled with his family to the USSR on the rise of Hitler; they were of course put into a labor camp. As both Jews and Communists they would hardly have survived in Berlin, so a Russian camp was a stark but better alternative to remaining in Nazi Germany. As a young girl Mrs. Possony had met Albert Einstein on a family visit to the United States, and from the USSR prison camp wrote him a letter addressed to "Dr. Albert Einstein, United States of America". The US Post Office delivered it to him at Princeton University. Einstein was gracious enough to reply, and even to send a small package of food and hygienic goods, which raised her status somewhat in Stalin's estimation. After Stefan's stroke she singlehandedly kept him alive for a decade when no one expected him to live a month.

Stefan T. Possony was a Senior Fellow of the Hoover Institution until he died. He had formerly been a Professor of Political Science at Georgetown University, and a Pentagon intelligence officer for the United States. Prior to the invasion of France in 1940 he was an intelligence officer in the French Air Ministry, to which he came from the Air Ministry of Czechoslovakia. His escape from France during the confusion of the Fall of France was a fascinating story; at one point he contemplated using a kayak to paddle to Spain, but managed to get one of the last tickets to Oran.

He had fled Czechoslovakia during the Nazi invasion. He had come to Prague from Vienna, where he obtained his Ph.D. at the University of Vienna and joined the Schusschnigg ministry opposing the Anschluss with Germany; he was on the Gestapo's wanted list, and left for Czechoslovakia as the Wehrmacht rode in. He used to say that the Gestapo got his library three times, in Vienna, Prague, and Paris. In the 70's and early 80's Stefan was quietly influential, directing several Pentagon studies of Soviet leadership and strategy. His biography of Lenin is still about the best tool for understanding the founder of the USSR. Alas it is long out of print.

Stefan Possony was perhaps the single most important member of my Citizens Advisory Council on National Space Policy which among other duties assisted Dr. Kane in writing Transition Team papers on space and military policy for the incoming Reagan Administration. Possony was one of the major architects of the Strategic Defense Initiative. Strategy of Technology introduced the notion of a strategy of "assured survival" in contrast to "assured destruction" and Assured Survival is the title of one chapter of that book.

Dr. Kane and I would like to revise the book and get it back in print, since the principles seem even more important now than they were when it was written. We're both getting old enough that we wonder if that will happen, but it should. As written it's still worth reading (in my judgment), and several War College students have used it as part of their advanced degree work. Revising it was going to be the project of one USAF officer at the post graduate school, but he was needed as director of a weapons lab and left the school before that could be done. Meanwhile, the book exists here.

Jerry Pournelle

Studio City, CA

Saturday, January 09, 1999

\begin{mdframed}[nobreak=true]
\begin{quotation}
"A gigantic technological race is in progress between interception and penetration and each time capacity for interception makes progress it is answered by a new advance in capacity for penetration. Thus a new form of strategy is developing in peacetime, a strategy of which the phrase ‘arms race’ used prior to the old great conflicts is hardly more than a faint reflection.
\newline
There are no battles in this strategy; each side is merely trying to outdo in performance the equipment of the other. It has been termed ‘logistic strategy’. Its tactics are industrial, technical, and financial. It is a form of indirect attrition; instead of destroying enemy resources, its object is to make them obsolete, thereby forcing on him an enormous expenditure…
\newline
A silent and apparently peaceful war is therefore in progress, but it could well be a war which of itself could be decisive."
\newline
--General d’Armee Andre Beaufre
\end{quotation}
\end{mdframed}

\section{Preface to the Electronic Edition 1997}
The quotation above opened the original edition of this book; it was clearly prophetic. The silent and apparently peaceful war was decisive.

This book was originally written in 1968 to 1970, a time when the Cold War was real and the outcome still very much in doubt; it will be recalled that Nixon’s Secretary of State Henry Kissinger, convinced that the Cold War was lost, hoped to negotiate détente and come to terms with Soviet International communism; and it was widely assumed in 1975 that the United States had been dealt a major defeat in Viet Nam.

In 1991, just before the collapse of the Soviet Union ended the Seventy Years War, we attempted to edit this work into a form suitable for publication in an electronic medium. This was well before the popularity of the world wide web, and before electronic publishing tools were readily available.

The end of the Seventy Years War brought other problems. The senior author, Dr. Stefan Possony, lived to see the victory which he had done so much to bring about, but died shortly after the collapse of international communism. Dr. Kane and Dr. Pournelle were involved in the development of the space program, and particularly the renewal of the X projects which had been canceled by McNamara in the name of Arms Control (because they were so successful at generating new military technology. New technology wasn’t wanted by those enamored of Arms Control strategies.)

For those and other reasons, this book languished for six years with little or no work done.

A generation of students used this book, but a new generation can’t find it; the copies still in use in the War College are Xeroxes, the book long being out of print. Meanwhile, new threats loom on the horizon. The Seventy Years War is over; the Technological War continues relentlessly. It is possible that this book is needed now more than ever.

Most of the examples in this book were chosen for their impact on thoughts about the Cold War and the threat of Soviet communism. They are now historical rather than current, and a proper revision of this book would use examples from current threats; alas we haven’t time to do that; nor have we time to do a proper chapter on space and space weapons. You will find THOR and SDI in these pages, but they aren’t given their proper emphasis. No matter. The principles in this book remain as true today as when they were written; we find little that needs explaining, and nothing that requires an apology.

Jerry Pournelle
Studio City, California 1997

\section{Preface to the Electronic Edition 1991}
When this book was originally published, the Cold War was very real. The United States was winding down the agony of Viet Nam, and one heard calls for "one, two, three, many Viet Nams" to bring the United States to her knees.

The threat of nuclear war was quite real, although it was not everywhere taken quite as seriously as it should have been.

The Soviet Union was not seen as an evil empire, but as the representative of the wave of the future.

The result was that the early portion of the book was devoted to convincing the readers that the threat was real, and imparting an understanding of the nature of that threat. That was needed then. It is less needed now; yet some of the early material also introduces the concepts of strategic analysis and the technological war, and those concepts are vital to understanding the principles we try to explain in this book.

A full rewrite of STRATEGY OF TECHNOLOGY would go through and pare away those portions written to respond to the threat of the 70's and would add new examples and analyses to fit the threat of the 90's. Alas, we have not time to do this; our choices are a 'quick fix' or not to publish for some years.

[That paragraph was itself written in 1991; what we did then was essentially nothing. It is clearly time to get this published in electronic form, whatever else we do.]

STRATEGY OF TECHNOLOGY was a textbook in the Service Academies for several years, and off and on has been a textbook in the Air and National Defense War Colleges. We have reason to believe that its arguments were useful in bringing about adoption of a high tech strategy for the US Armed Forces. That such a strategy was adopted is self evident from the victory in Iraq and the collapse of the Soviet empire. How much was due to this book can be debated, but we can at least claim that this book explains the principles of technological strategy.

Some day we will revise the examples. However, the principles haven't changed, and the rapid changes in the Soviet Union as well as the Iraq victory can be explained as consequences of an earlier victory in the 'silent and apparently peaceful conflict which may be decisive' which we called The Technological War.

From time to time we have inserted comments made at times later than the first publication. Those are marked with brackets and dated. We find we haven't had to do much revision of the book, and none of the principles espoused needed changing. We have pointed up new examples of the application of those principles.

Portions of this revised text have from time to time been published in different volumes of THERE WILL BE WAR, an anthology series edited by Jerry Pournelle.
